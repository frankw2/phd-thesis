\chapter{Related Work}
\label{sec:related}

Sieve is related to prior work in
several areas: using cryptography to protect data on untrusted
servers; using ABE to protect storage; using oblivious RAM; and other
delegation protocols.  The rest of this chapter discusses these areas
of related work in turn.

\section{Untrusted servers} 
Browser extensions
like ShadowCrypt~\cite{shadowCrypt} transparently
encrypt the data that a browser sends to
unmodified cloud servers. Intentionally
encrypted cloud stores like SUNDR~\cite{sundr},
Depot~\cite{depot}, and SPORC~\cite{sporc}
provide stronger consistency semantics in the
face of server-side misbehavior; application
logic runs solely on the client-side, over
cleartext data, with clients exchanging encrypted
data with servers. 
Other systems that store encrypted data on servers
and run application logic on the client-side
include BStore~\cite{bstore} and DepSky~\cite{depSky}.
All of these systems prevent
data leakage due to server compromise or malice,
but they are incompatible with applications that
leverage server-side computation to add value to
raw user data. In contrast, Sieve 
is totally compatible with server-side computation.

In CryptDB~\cite{cryptdb}, a web application
consists of clients, an application server,
and a back-end database. The database only
contains encrypted data. Using SQL-aware encryption,
the application server can execute queries over
the encrypted data without revealing cleartext
to the database. However, the application server
does see cleartext, and can leak user data if
compromised. Mylar~\cite{mylar} eliminates the
need for an application server, but restricts
the encrypted server-side computation to keyword
searches. In both CryptDB and Mylar, applications
control how user data is shared. In Sieve, user
data is decoupled from applications, and users
can selectively disclose individual objects to
third parties.

% Here's a semi-comprehensible slide deck about
% Privly:
%    https://github.com/privly/privly-organization/blob/master/presentations/2013-07-25-OSCON/OSCON.pdf
% All of the other stuff that I've found is
% even more vague and hand-wavy. If we're running
% out of space in the paper, we should cut
% the Privly discussion.
Privly~\cite{privly} allows users to upload
encrypted data to a storage server, and share
hyperlinks to that data. The hyperlinks can be
embedded in sites like a Facebook page, but
the hyperlinks reveal no cleartext to the
owner of the embedding site. Users register
their decryption keys with Privly's browser
extension. Later, when the user visits a page
and her extension finds a Privly hyperlink,
the extension transparently fetches the
encrypted data, decrypts it, and rewrites
the page's HTML, replacing the Privly link
with the cleartext data. Privly does not supports
the server-side computation enabled
by Sieve.

\section{ABE-protected storage} 
Persona~\cite{persona},
Priv.io~\cite{privio}, and Cachet~\cite{cachet}
use ABE to selectively expose encrypted user data.
In Persona and Priv.io, each user keeps her data
in private cloud storage; in Cachet, data is
stored in a peer-to-peer, distributed hash table.
Unlike Sieve, these systems cannot delegate access
to third party services, breaking compatibility
with rich applications that require server-side
computation.
% Persona, Priv.io, and Cachet also lack Sieve's
% standardized data schemas (\S\ref{sec:attrGen}).
% Thus, these systems do not support multi-service
% interoperability (i.e., users cannot write data
% once, in a well-known format, and have that data
% be seamlessly consumed by multiple third parties).
Persona, Priv.io, and Cachet also trust each device
for the  lifetime of the system, whereas Sieve
distributes this trust. Finally, Sieve provides a
concrete revocation protocol that safeguards user
data if storage servers are compromised. Priv.io
has no revocation strategy, and Persona suggests
re-keying data, but does not provide a specific
protocol. Cachet does implement revocation, but
requires a trusted proxy which must interpose on
all decryption operations, even in the common case
that revocation is not underway~\cite{easier}.
Cachet's revocation scheme also does not re-encrypt
data on storage providers, leaving data encrypted
under revoked keys vulnerable to subsequent
compromises of the storage provider.

\section{Predicate encrypted storage}
GORAM~\cite{GORAM} is an oblivious RAM system
that allows users to selectively share their
cloud data with other users. Clients place
encrypted data on servers so that servers cannot
inspect it, and the clients' access patterns are
hidden from servers using ORAM shuffling
techniques~\cite{shroud}. Like Sieve, GORAM tags
data objects with attributes; unlike Sieve,
GORAM uses attribute-hiding predicate encryption~\cite{katz2008,shen2009}
to restrict knowledge of attributes to the
clients that possess matching decryption keys.

GORAM's use of oblivious RAM and predicate
encryption provides stronger security than Sieve,
but at a performance cost. To hide
data access patterns from storage servers, GORAM
clients must perform \texttt{O(polylog(n))} more
data accesses. Hiding attribute values using
predicate encryption substantially increases 
GORAM's ciphertext size and its encryption
and decryption times. In comparison, Sieve
provides lower-latency web services with less
user friction. GORAM also forces users to
determine a priori the maximum number of
principals that can be mentioned in access
control lists; if this list changes, a user must
re-initialize her database. GORAM has no revocation
scheme and assumes that users have one trusted
device whereas Sieve provides a complete revocation
scheme and prevents against device losses
that might lead to a compromise of the whole
system.

\section{Access delegation schemes}

%% aauth uses time slots for revocation

OAuth~\cite{oauth} is a widely-used protocol
for sharing cloud data across different web
services. OAuth policies are written by web
services, not by users, so users lack true
authority over their access controls. OAuth
also does not leverage cryptography to protect
user storage or enforce access policies. As a
result, users have no strong assurances about
how their data is exposed. OAuth is
also vulnerable to various kinds of data
leaks~\cite{oauthbugs, oauthbugs2}.
AAuth~\cite{aauth} is an extension of
OAuth which uses cryptography
to delegate access to encrypted data,
but relies on the existence
of various trusted parties to enforce
access policies. In Sieve, users
generate their own policies and distrust
the storage server and third party
applications. Sieve's protocols are 
simpler, and Sieve's policy
language is richer than AAuth's
fixed policy schemas. AAuth's revocation scheme,
like Persona's, only revokes access
to the metadata block, but consider
cached symmetric keys
that might leak cleartext data when
the cloud provider is compromised.

The OAuth protocol generates a token
that principals use to access sensitive
data. Web services define many other types
of ``bearer tokens.'' HTTP cookies~\cite{cookies}
are a classic example. Macaroons~\cite{macaroons}
improve upon cookies, using chained HMACs
to verify and attenuate capabilities as a
macaroon is passed between multiple parties.
Cookies and macaroons vouch for a principal's
post-authorization status, whereas Sieve
deals with the authorization itself.
