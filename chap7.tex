\chapter{Conclusion and Future Work}

\section{Conclusion}

Sieve is a new access control system
that allows users to selectively expose
their private cloud data to third party
web services. Sieve leverages attribute-based
encryption to translate human-understandable
access policies into cryptographically
enforceable restrictions, and
is the first ABE system to protect against
device loss and supports full revocation
of both data and metadata. Unlike prior
solutions for encrypted storage, Sieve is
compatible with rich, legacy web applications
that require server-side computation.
As a proof of concept, we integrated Sieve
with an open-source health monitoring
service, showing that Sieve is a
practical mechanism for protecting
user data.

\section{Future Work}

There a number of interesting directions
to follow up on Sieve. Sieve provides
an access control mechanism for user-managed
data for web applications. An interesting
question is how can users interact with 
each other in a peer-to-peer setting. 
For example, can users share data or do 
computations on each other's data
revealing minimal data? Another interesting
question is whether users can participate
in surveys without revealing to the web
application their value. For instance,
assume Amazon wants to know how many
people purchased an item. Can users
release their data such that Amazon
only learns the sum but not any
individual answers.

Finally, another area of interesting work
involves Sieve user data after it has
been released to web applications.
Is there a way to control the flow
of information out of applications
based on ABE-policies? Is there a
way to extend Sieve to regulate
the export of data between applications
and maybe other users.

Sieve has envisioned a new web ecosystem
that is beneficial for both applications
and users. It is important to see how
Sieve can be applied to other important
web applications.