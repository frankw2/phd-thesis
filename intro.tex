\section{Introduction}
\label{chap:intro}

\subsection{Motivation}
Consumers are increasing their usage of web services. Whenever
users interact with these applications, the web service collects
their data, which can contain sensitive 
information, such as their medical conditions,
political preferences, and income~\cite{narayanan2010myths, narayanan2008robust}.
Unfortunately, this data can 
leak through numerous channels on both the server
and client. As web systems become more complex, the 
number of these channels continues to grow. More specifically,
on the server side, data can leak through data breaches
or be accessed by malicious service administrators. On the client side,
sensitive user information can leak from the browser whenever
a user accesses a web application. These scenarios raise an important
question. How do we build practical, secure
mechanisms to prevent this data leakage?

Much research has focused on building systems~\cite{popa:mylar, popa:cryptdb, li:sundr, feldman:sporc} 
that protect sensitive user data stored in web application databases
or cloud providers
from leaking as a result of data breaches or malicious service
providers. However, the focus of
this dissertation is different. We are not protecting sensitive 
user data that already exists on the server.
Instead, we build systems
that prevent data leakage on the client (Veil), specifically
from the browser, and that protect sensitive data in queries (Splinter).
In the section below, we provide an overview of these two systems.

\subsection{Our Systems}

\subsubsection{Veil: Private Browsing Semantics without Browser-side Assistance}
All popular web browsers offer a ``private browsing
mode.'' After a private session terminates, the
browser is supposed to remove client-side
evidence that the session occurred. Unfortunately,
browsers still leak information through the file
system, the browser cache, the DNS cache, and
on-disk reflections of RAM such as the swap file.

Veil is a new deployment framework that allows
web developers to prevent these information leaks,
or at least reduce their likelihood. Veil leverages
the fact that, even though developers do not control the
client-side browser implementation, developers do
control 1) the content that is sent to those browsers,
and 2) the servers which deliver that content.
Veil web sites collectively store their content
on Veil's \emph{blinding servers} instead
of on individual, site-specific servers. To publish
a new page, developers pass their HTML, CSS, and
JavaScript files to Veil's compiler; the compiler
transforms the URLs in the content so that, when
the page loads on a user's browser, URLs are derived
from a secret user key. The blinding service and
the Veil page exchange encrypted
data that is also protected by the user's key. The
result is that Veil pages can safely store encrypted
content in the browser cache; furthermore, the URLs exposed
to system interfaces like the DNS cache are
unintelligible to attackers who do not possess
the user's key. To protect against post-session
inspection of swap file artifacts, Veil uses
heap walking (which minimizes the likelihood that
secret data is paged out), content mutation (which
garbles in-memory artifacts if they do get swapped out),
and DOM hiding (which prevents the browser from
learning site-specific HTML, CSS, and JavaScript
content in the first place). Veil pages load on unmodified
commodity browsers, allowing developers to provide
stronger semantics for private browsing without forcing
users to install or reconfigure their machines. Veil
provides these guarantees even if the user does not
visit a page using a browser's native privacy mode;
indeed, Veil's protections are \emph{stronger}
than what the browser alone can provide.

\subsubsection{Splinter: Practical, Private Web Application Queries}

Many online services let users query datasets such as maps, flight prices, patents,
and medical information. The datasets themselves do not contain sensitive user information,
but unfortunately, users' queries on these datasets can reveal sensitive information.
This dissertation presents Splinter, a system that protects users' queries and
scales to realistic applications.
A user splits her query into multiple parts and sends each part 
to a different provider that holds a copy of the data.
As long as any one of the providers is honest and does not collude with the
others, the providers cannot determine the query.
Splinter uses and extends a new cryptographic primitive called Function Secret Sharing (FSS) 
that makes it up to an order of magnitude more efficient than prior systems based on 
other cryptographic techniques such as Private Information Retrieval and garbled circuits.
We develop protocols extending FSS to new types of queries, such as MAX and TOPK queries. 
We also provide an optimized implementation of FSS using AES-NI instructions and multicores.
Splinter achieves end-to-end latencies below 1.6 seconds for realistic workloads 
including a Yelp clone, flight search, and map routing.

%\subsubsection{Dissertation Roadmap}
%
%The dissertation will be organization like the following: 
%Chapter~\ref{chap:veil} and Chapter~\ref{chap:splinter} will motivate and describe Veil
%and Splinter respectively. Chapter~\ref{chap:concl} will describe future work.
