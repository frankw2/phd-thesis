\section{Certifying FSCQ's logging system}
\label{sec:log}

\newcommand{\logapi}{LogAPI\xspace}
\newcommand{\grouplog}{GroupCommit\xspace}
\newcommand{\memlog}{Applier\xspace}
\newcommand{\disklog}{DiskLog\xspace}

The previous chapters provide examples of specifications for a basic
logging protocol, written in \chl.  This chapter explores how to build and
certify \syslog, an I/O-efficient logging system that implements
performance-critical optimizations such as deferred apply, group commit,
log checksum and log bypass.  These techniques, as we explained in
\Cref{sec:bkgrd:log}, reduce the number of writes and write barriers per
transaction, while still guaranteeing crash safety.  These optimizations
are standard to other file systems but have never been formally
specified or verified until now.  The \syslog implements these
optimizations internally, while providing the same synchronous disk
abstraction as in previous chapters.

\subsection{Overview}

Verifying a single optimization is challenging enough because most system
optimizations are based on a programmer's assumptions, which are rarely
formalized. Verifying many together is even more challenging because of the
number of simultaneously moving parts. Each additional optimization
requires not only a proof that the overall logging specification still
holds, but also a proof that all previous assumptions still hold.
For example, group commit assumes that metadata operations can be deferred
as long as their orders are preserved, and log bypass further allows
deferring and re-ordering non-metadata updates; combining the two to
support a general deferred-write optimization poses a new challenge: is it
safe to re-order non-metadata updates against metadata updates?

The key idea behind verifying a system with many optimizations is one
familiar from building the unverified equivalent: modularity.  \syslog is
composed of four logical layers: \logapi, \grouplog, \memlog, and \disklog,
shown in \Cref{fig:log}.  Each layer has a formal specification, along with
a proof that its implementation meets its specification.  This modular
design makes \syslog amenable to formal verification because each layer
only tackles a simple task or a single optimization.  It also allows
changing the implementation of a layer without affecting the proofs in
other layers.

\begin{figure}[htb]
  \centering
  \includegraphics[width=0.9\textwidth]{figs/log.pdf}
  \caption[Illustration of \syslog layers and the timeline of a transaction]
  { Illustration of \syslog layers and the timeline of a transaction.
  Transactional writes are added to an \V{activeTxn} map in the \logapi
  layer.  When the application calls \cc{commit}, \V{activeTxn} is buffered
  in a list of pending transactions in \grouplog. When \cc{flush} is called
  on \grouplog, all pending transactions are appended to disk together as a
  single transaction in the \disklog layer. At this point, the transactions
  are durable, and \memlog can lazily apply and truncate the log records.}
  \label{fig:log}
\end{figure}

\paragraph{\logapi.}

The uppermost layer, \logapi, exposes an interface with a single active
transaction and allows the higher-level code (i.e., the file system) to
read and write disk blocks.  Writing blocks builds up an in-memory
transaction, which is passed to the \grouplog layer once the higher-level
code invokes \cc{log\_commit}.  \logapi exposes the size of the transaction
in its specification and guarantees that transactions below a certain size
will be able to commit.  This is necessary for proving that some system
calls that potentially update many disk blocks (e.g., \cc{unlink} might
change all block-bitmap blocks when freeing a large file) do not fail given
sufficient log space.

\paragraph{\grouplog.}

\grouplog accepts committed transactions from \logapi and buffers them in
an in-memory transaction list.  \grouplog exposes a \cc{flush} function,
which combines all buffered transaction into a single transaction and then
flushes it to the on-disk log.  \cc{flush} allows the file system to
implement the \cc{fsync} system call by flushing all metadata changes from
\grouplog to disk.  \grouplog's specification also allows it to flush
transactions to disk on its own at any time.  \grouplog remembers
boundaries of transactions it receives. In case the merged transaction
exceeds the maximum transaction size, \grouplog falls back to committing
individual transactions in turn.

\paragraph{\memlog.}

\memlog manages the data part of the disk (i.e., everything but the log) by
applying the log entries to the disk when the on-disk log fills up.  By
employing deferred apply, \memlog is able to absorb
repeated writes to the same address in multiple transactions. To do this,
\memlog keeps a copy of all flushed entries in memory when forwarding
flushing requests from \grouplog to \disklog.

\paragraph{\disklog.}

Finally, \disklog lays out the log on disk.  It provides three functions:
\cc{append}, \cc{truncate}, and \cc{recover} (which returns the on-disk log
contents).  It takes care of packing and unpacking the commit block and log
entries on disk.  \disklog also exposes the size of the on-disk log to
guarantee that transactions of a certain size will fit and thus be able to
commit.  \disklog uses log checksum to ensure that \cc{append} and
\cc{truncate} are crash-safe while using just one disk sync (to ensure that
the entire transaction is flushed to disk, rather than to order writes
within the transaction).  \disklog's \cc{recover} function also uses
checksums to determine whether the last call to \cc{append} fully made it
to disk or not, and returns the corresponding list of on-disk log entries
to the caller.

Each layer (except for \disklog) also caches copies of transaction updates
in an in-memory map to serve reads efficiently. As shown in \Cref{fig:log},
\logapi buffers the file system's single active transaction
(\V{activeTxn}). \grouplog merges each committed transaction that it
receives into a single in-memory transaction (\V{committedTxns}), which is
cleared at every flush. Similarly, \memlog collapses all flushed
transactions into an in-memory map that is cleared every time the log is
applied and truncated (\V{flushedTxns}).  When an application calls
\cc{read}, the log tries to find the requested address in each of these
layers, from highest to lowest, before reading from disk.

\subsection{Representation invariants}

As described in \Cref{s:chl:ns}, \syslog's specifications use
representation invariant to relate the abstract state of a higher layer to
the its lower-layer state.  For example, \disklog defines
\cc{disklog\_rep}, which describes how logical log entries (as a list of
block address and value pairs) are laid out on the log region of the disk.
\memlog, in turn, includes \cc{disklog\_rep} as part of its own
\cc{applier\_rep}, and combines it with other points-to facts that
relate the logical log entries to the data region of the disk.  In its
easiest form, where there is no ongoing log-apply or log-flush, the
definition of \cc{applier\_rep} is shown in \Cref{fig:applier_rep}.

\begin{figure}[htb]
\centering
\specfont
\begin{equation*}
\begin{aligned}
  \F{applier\_rep}\,
  & (\,\V{navail}, ~\V{flushed\_disk})~\defeq \\
  & \F{disklog\_rep}(\C{Synced},~\V{navail},~\V{flushedTxns}) * \\
     &(\forall a, ~\V{unapplied\_disk}(a) = \valuset{v}{vs} \; =>\;
          a |+> \valuset{v}{vs}) \AND \\
  & \V{flushed\_disk} = \F{replay}\,(\V{unapplied\_disk}, ~\V{flushedTxns})
\end{aligned}
\end{equation*}
\caption{Part of \memlog's representation invariant}
\label{fig:applier_rep}
\end{figure}

\cc{applier\_rep} states that the physical disk is divided into two
disjoint regions: the log region and the data region. \cc{disklog\_rep}
describes the state of the log region, which contains all flushed but
not-yet-applied transactions, represented by \V{flushed\_Txns}; and
\V{unapplied\_disk} is the on-disk state of the data region.  By replaying
\V{flushed\_Txns} on \V{unapplied\_disk}, \cc{applier\_rep} exports a new
logical disk \V{flushed\_disk} in its argument, which combines the on-disk
data region with all flushed transactions to present a view of the
persisted disk state.  The other argument of \cc{applier\_rep}, \V{navail},
is the number of available log entries exported by \disklog; it is useful
for higher layers to reason about whether a commit or a flush would
succeed.

Similarly, as shown in \Cref{fig:group_rep}, \grouplog's representation
invariant \cc{group\_rep} uses \cc{applier\_rep} to describe the disk state
which currently persists on disk.  On top of it, \cc{group\_rep} describes
a sequence of logical disks, namely \V{disk\_seq}, each representing the
end state of a committed but not-yet-flushed transaction.  The definition
in \Cref{fig:group_rep} assumes the first disk (i.e., {\V{disk\_seq}[0]})
is persisted on disk, and each subsequent disk in \V{disk\_seq} is derived
from its predecessor by applying the corresponding committed transaction
from \V{committedTxns}.  The abstraction exported by \cc{group\_rep} is
called a \emph{disk sequence}. In \Cref{sec:spec} we further justify the
choice of the disk sequence abstraction and investigate how it helps to
write a file system's specifications.

\begin{figure}[htb]
\centering
\specfont
\begin{equation*}
\begin{aligned}
  \F{group\_rep}\, & (\,\V{disk\_seq})~\defeq \\
  & \F{applier\_rep}(\V{navail},~\V{disk\_seq}[0]) \AND \\
  & (\forall i,\; 0 < i < \len{\V{disk\_seq}} => \\
  & \qquad \len{\V{committedTxns}[i-1]} \leq \C{MaxLogLen} \AND \\
  & \qquad \V{disk\_seq}[i] = 
    \F{replay}\,(\V{disk\_seq}[i-1],~\V{committedTxns}[i-1]))
\end{aligned}
\end{equation*}
\caption{Part of \grouplog's representation invariant}
\label{fig:group_rep}
\end{figure}

\Cref{fig:log_rep_new} shows the top-level representation invariant
\cc{log\_rep}, whose signature looks almost identical to the one shown in
\Cref{fig:logrep}, except that it now uses a disk sequence, as opposed to a
single logical disk, to describe the transaction's starting state.
``$\latest{disk\_seq}$'' denotes the last disk in the disk sequence;
applying the active transaction's log entries (\V{activeTxn}) to the latest
disk produces \V{cur\_state}, the logical disk that represents the
transaction's current view.  If there is no active transaction,
\cc{log\_rep} requires \V{activeTxn} to be empty.

Note that most of the above representation invariants have more
sophisticated forms which show up only during crash and recovery.  For
simplicity we do not show them here but will explain them as we encounter
them.

\begin{figure}[htb]
\centering
\specfont
\begin{equation*}
\begin{aligned}
  \F{log\_rep} \, & (\C{ActiveTxn}, ~\V{disk\_seq}, ~\V{cur\_state})~\defeq \\
  & \F{group\_rep}(\V{disk\_seq}) \AND \\
  & \V{cur\_state} = \F{replay}\,(\latest{disk\_seq}, ~\V{activeTxn})
  &\\
  \F{log\_rep} \, & (\C{NoTxn}, ~\V{disk\_seq})~\defeq \\
  & \F{group\_rep}(\V{disk\_seq}) \AND \V{activeTxn} = \empset
\end{aligned}
\end{equation*}
\caption{Part of \logapi's representation invariant}
\label{fig:log_rep_new}
\end{figure}


\subsection{Logging-system specifications}

Given these representation invariants, we now demonstrate how to use them
to write \syslog's internal and external specifications.  We consider an
interesting scenario where the user of \syslog (i.e., the file system) uses
\cc{log\_commit} to commit a transaction and subsequently calls
\cc{log\_flush} to make sure the change persists.  The call graph used
in this section is shown in \Cref{fig:log-call}.

\begin{figure}[htb]
  \centering
  \includegraphics[width=0.9\textwidth]{figs/logcall.pdf}
  \caption{Call graph for \cc{log\_commit} and \cc{log\_flush}}
  \label{fig:log-call}
\end{figure}

\Cref{fig:spec_log_commit} shows the specification of \cc{log\_commit}.
The spec says that \cc{log\_commit} transits from ``active transaction''
state to ``no transaction'' state, and it can either succeed (return true)
or fail (return false).  If it succeeds, the logical disk for the active
transaction \V{cur\_disk} is appended to the disk sequence \V{disk\_seq};
if it fails, the transaction is aborted, and \V{disk\_seq} will remain
unchanged.  In addition, the specification also says that if commit fails,
it must be the case that the size of the transaction (total number of log
entries in \V{activeTxn}) exceeds the logging system's limit, denoted by
constant \C{MaxLogLen}.  This error condition allows the caller to prove
that certain operations (such as \cc{unlink}) will always succeed (by
showing that $\len{\V{activeTxn}} \le \C{MaxLogLen}$, thus proving
that the failure case in the postcondition cannot happen).

\begin{figure}[htb]
\begin{spec}
    \SPEC  {\PROC{log\_commit}{}}
    \PRE   {\PRED{disk}{\F{log\_rep}(\C{ActiveTxn}, \V{disk\_seq}, \V{cur\_disk})}}
    \POST  {\PRED{disk}{%
            $ ( \V{ret} = \true\;\AND 
                \F{log\_rep}(\C{NoTxn},~\cons{cur\_disk}{disk\_seq})) \OR $ \BR
            $ ( \V{ret} = \false \AND
                \F{log\_rep}(\C{NoTxn},~\V{disk\_seq})
                \AND \len{\V{activeTxn}} > \C{MaxLogLen}) $
            }}
    \CRASH {\PRED{disk}{\F{log\_rep}(\C{NoTxn}, \V{disk\_seq})}}
\end{spec}
\caption{Specification for \logapi's \cc{log\_commit}}
\label{fig:spec_log_commit}
\end{figure}

Internally, \cc{log\_commit} calls \grouplog's
\cc{group\_commit}(\V{activeTxn}), which buffers the transaction in memory.
The specification of \cc{group\_commit} is shown in
\Cref{fig:spec_group_commit}.  This specification looks similar to
\Cref{fig:spec_log_commit}, except that it expands \V{cur\_disk}, using the
fact that directly derives from \logapi's representation
invariant~(\Cref{fig:log_rep_new}). As \cc{group\_commit} only changes the
in-memory state, its crash condition is the same as its precondition.

\begin{figure}[htb]
\begin{spec}
    \SPEC  {\PROC{group\_commit}{ents}}
    \PRE   {\PRED{disk}{\F{group\_rep}(\V{disk\_seq})}}
    \POST  {\PRED{disk}{%
            $ ( \V{ret} = \true~\AND \F{group\_rep}(\cons{\F{replay}\,(
               \latest{disk\_seq},~\V{ents})}{disk\_seq} \OR $ \BR
            $ ( \V{ret} = \false \AND \F{group\_rep}(\V{disk\_seq}) \AND
                \len{\V{ents}} > \C{MaxLogLen}~) $
            }}
    \CRASH {\PRED{disk}{\F{group\_rep}(\V{disk\_seq})}}
\end{spec}
\caption{Specification for \grouplog's \cc{group\_commit}}
\label{fig:spec_group_commit}
\end{figure}

Next, the user calls the top-level \cc{log\_flush}, whose specification is
shown in \Cref{fig:spec_log_flush}.  The specification simply says that,
after \cc{log\_flush} returns, the disk sequence contains only the latest
disk from before \cc{log\_flush}. This latest disk reflects all of the
previously committed transactions.

One interesting aspect of \cc{log\_flush}'s spec is the crash condition:
``\cc{would\_recover\_any} (\V{disk\_seq})'' says that if the system were
to crash, the state on disk after the crash is the state corresponding to
one of the disks in \V{disk\_seq}.  This is because \cc{log\_flush}
internally calls \grouplog's flush procedure \cc{group\_flush}, which in
turn calls \memlog's \cc{applier\_flush}, with the merged transaction as
the argument (i.e.,
\cc{applier\_flush}(\cc{merge}(\V{committedTxns}))). In this case, the
logging system will recover into either the first or the last disk in
\V{disk\_seq}.  It is also possible that \grouplog cannot merge the
transactions (because the resulting transaction is too large to fit into
the log) and falls back to flush each transactions in turn using
\cc{applier\_flush}.  If this is the case, the logging system can recover
into any of the disks in \V{disk\_seq}.  Both cases are captured by
\cc{would\_recover\_any}.

\begin{figure}[htb]
\begin{spec}
    \SPEC  {\PROC{log\_flush}{}}
    \PRE   {\PRED{disk}{\F{log\_rep}(\C{NoTxn},~\V{disk\_seq})}}
    \POST  {\PRED{disk}{\F{log\_rep}(\C{NoTxn},~\{\latest{disk\_seq}\})}}
    \CRASH {\PRED{disk}{\F{would\_recover\_any}(\V{disk\_seq})}}
\end{spec}
\caption{Specification for \logapi's \cc{log\_flush}}
\label{fig:spec_log_flush}
\end{figure}

The specification of \cc{applier\_flush} is shown in
\Cref{fig:spec_applier_flush}, which also contains a success case and a
failure case.  However, the \grouplog layer can conclude that the failure
case will never happen.  This is because (1) the code of \cc{group\_flush}
first checks if the merged transaction is bigger than \cc{MaxLogLen} and
falls back to flushing individual transactions in that case; and (2)
\cc{group\_commit} will reject a transaction whose size is bigger than
\cc{MaxLogLen}, so that every buffered transaction is within the size
limit. The second constraint is encoded in the representation invariant
shown in \Cref{fig:group_rep}.

The crash condition of \cc{applier\_flush} is a
``\cc{would\_recover\_either}'' predicate, which says that the procedure
will recover into either before or after the given transaction---the usual
behavior of basic write-ahead logging.  The ``\cc{would\_recover\_any}''
predicate we have seen in \Cref{fig:spec_log_flush} is defined using
\cc{would\_recover\_either} by choosing pairwise disks from \V{disk\_seq}.

\begin{figure}[htb]
\begin{spec}
    \SPEC  {\PROC{applier\_flush}{ents}}
    \PRE   {\PRED{disk}{\F{applier\_rep}(\V{navail},~\V{flushed\_disk})}}
    \POST  {\PRED{disk}{%
            $ ( \V{ret} = \true\;\AND \F{applier\_rep}
                (\V{navail}',~\F{replay}\,(\V{flushed\_disk},~\V{ents}))) \OR $ \BR
            $ ( \V{ret} = \false \AND \F{applier\_rep}
                (\V{navail},~\V{flushed\_disk}) \AND $ \BR
            $ \qquad\qquad\qquad  \len{\V{ents}} > \C{MaxLogLen}) $
            }}
    \CRASH {\PRED{disk}{\F{would\_recover\_either}(\V{flushed\_disk},
            \F{replay}(\V{flushed\_disk},~\V{ents}))}}
\end{spec}
\caption{Specification for \memlog's \cc{applier\_flush}}
\label{fig:spec_applier_flush}
\end{figure}

Inside \cc{applier\_flush}, if it sees the available log space \V{navail}
is too small to fit the passed-in transaction~(\V{ents}), the code will first
invoke \cc{applier\_apply} to make room for the new transaction.
\cc{applier\_apply} applies and truncates the on-disk log. Its
specification is shown in \Cref{fig:spec_applier_apply}.  The postcondition
of \cc{applier\_apply} resets the available log space to its maximum value.
Other than that, the specification exhibits a no-op-like behavior such that
\memlog is free to invoke it anytime. 

Also note that the crash condition in \Cref{fig:spec_applier_apply} is
different from its precondition. This is because \cc{applier\_apply} will
call \cc{disklog\_truncate} to clear the log, and \cc{disklog\_truncate}
might leave the log in an \emph{unsynced} state, in which \disklog just
overwrites the commit block but has not yet issued a write barrier to
persist the change.  The unsynced state is not captured by the regular form
of \cc{applier\_rep} shown in \Cref{fig:applier_rep}.  Nevertheless, this
does not invalidate the fact that applying the log is an idempotent
operation and will always recover to \V{flushed\_disk} as in the
precondition. The ``\cc{would\_recover\_before}'' predicate exactly
captures this property.

\begin{figure}[htb]
\begin{spec}
    \SPEC  {\PROC{applier\_apply}{}}
    \PRE   {\PRED{disk}{\F{applier\_rep}(\V{navail},~~\V{flushed\_disk})}}
    \POST  {\PRED{disk}{\F{applier\_rep}(\C{MaxLogLen},~~\V{flushed\_disk})}}
    \CRASH {\PRED{disk}{\F{would\_recover\_before}(\V{flushed\_disk})}}
\end{spec}
\caption{Specification for \memlog's \cc{applier\_apply}}
\label{fig:spec_applier_apply}
\end{figure}

Finally, \cc{applier\_flush} extends the log using \disklog's
\cc{disklog\_append}, whose specification is illustrated in
\Cref{fig:spec_disklog_append}. Like before, \syslog's implementation
guarantees that the failure case in the postcondition cannot occur, because
no transaction larger than \cc{MaxLogLen} will be appended, and
\cc{applier\_apply} will reset the available log space back to
\cc{MaxLogLen} when necessary.  

The two branches in \Cref{fig:spec_disklog_append}'s crash condition
corresponds to the two cases defined by \cc{would\_recover\_either}.  In
particular, the ``\cc{Extended}'' case also describes an unsynced log
state: If the system crashes while updating the commit block, it will
recover to a state either before or after \V{new\_ents} is appended.  The
evolution of crash conditions in each logging layer further demonstrates
that representation invariants can help in hiding lower-level details from
the higher layers.

One subtlety of \cc{disklog\_append}'s specification is that it includes a
frame predicate $F_{\V{disk}}$.  This is because \cc{disklog\_rep} only
describes the log region and does not cover the entire \V{disk} address
space. This frame predicate allows \disklog's upper layer (\memlog) to
carry its own predicate about the data region of the disk---another example
of using separation logic to achieve modularity.

\begin{figure}[htb]
\begin{spec}
    \SPEC  {\PROC{disklog\_append}{new\_ents}}
    \PRE   {\PRED{disk}{$ F_{\V{disk}} * \F{disklog\_rep}
                (\C{Synced},~\V{navail},~\V{old\_ents})$}}
    \POST  {\PRED{disk}{%
            $ ( \V{ret} = \true\;\AND F_{\V{disk}} * \F{disklog\_rep}
                (\C{Synced},~\V{navail}',~\V{old\_ents \app\,new\_ents}) $ \BR
            $ ( \V{ret} = \false \AND F_{\V{disk}} * \F{disklog\_rep}
                (\C{Synced},~\V{navail},~\V{old\_ents}) \AND $ \BR
            $ \qquad\qquad\qquad\;~
              \len{\V{new\_ents}} > \V{navail}\,) $
            }}
    \CRASH {\PRED{disk}{%
           $ F_{\V{disk}} * \F{disklog\_rep} 
                (\C{Synced},~\V{navail},~\V{old\_ents}) ~~ \OR $  \BR
           $ F_{\V{disk}} * \F{disklog\_rep}
                (\C{Extended},~\V{old\_ents},~\V{new\_ents})$}}
\end{spec}
\caption{Specification for \disklog's \cc{disklog\_append}}
\label{fig:spec_disklog_append}
\end{figure}



\subsection{Logging with checksums}

The previous section mostly introduced specifications above the \disklog
layer that implement deferred apply and group commit.  This section
describes how to implement and formalize the log-checksum optimization
mentioned in \Cref{sec:bkgrd:log}.  More details about the design of \sys's
log-checksum optimization can be found in~\cite{stephanie-meng}.

%Traditional write-ahead logging protocols use two write barriers in order
%to commit a transaction.  The protocol must first write the log entries
%to disk, and then issue a write barrier to ensure all log entries are
%persistently stored on disk.  The protocol can then write the commit
%block, indicating that the transaction has committed, and then issue a
%second write barrier to ensure that the commit record is durably stored
%before returning to the application code.  For instance, the Linux ext3
%file system uses this logging protocol for its crash recovery.
%
%Write barriers are expensive, and one way to reduce the number of
%write barriers is to rely on \emph{checksums} in the logging protocol.
%In particular, if the commit block contains a checksum of all the
%related log entries, it is safe to omit the write barrier between the
%log entries and the commit block.  The reason that it is safe is that,
%during recovery, the logging system can determine whether the log entries
%made it to disk, by reading the log entries from disk, computing their
%checksum, and comparing it to the checksum in the commit block.  If they
%match, then (with overwhelming probability) the log entries are correct,
%and the transaction can commit.  If they do not match, then the recovery
%code aborts the transaction.  This protocol is implemented as one option
%in the ext4 file system on Linux, and it is able to commit a transaction
%with just one write barrier as opposed to two.

\subsubsection{On-disk layout and protocol}
\label{sec:log:layout}

\disklog is responsible for managing on-disk state of transactions. The
on-disk log that \disklog maintains internally is separated into three
regions: the header (i.e., the commit block), descriptor, and data regions,
shown in \Cref{fig:loglayout}. The header stores the number and checksum of
valid blocks in the descriptor and data regions. To ensure durability after
a crash, the header also stores the number of blocks at the time of the
previous flush. The descriptor region stores disk-block addresses
corresponding to block value updates, which are stored in the data region
in the same order.  Disk-block addresses from a single \cc{append} call can
be packed into a single descriptor block.  \disklog's specifications show
that the packing and unpacking are sound.

\begin{figure}[htb]
    \centering
    \includegraphics[width=\textwidth]{figs/disklog.pdf}
    \caption{On-disk layout of \syslog}
    \label{fig:loglayout}
\end{figure}

To truncate the log, \disklog simply sets the length equal to zero in the
log header.  To append to the log, \disklog writes log entries and the log
header together. To do this, \disklog first writes each entry in the
transaction to the descriptor and data regions. Then, \disklog updates the
\V{checksum}, \V{previous\_len}, and \V{len} fields in the log header.  The
checksum is computed by hashing the stored checksum with the newly appended
descriptor and data blocks.  For both appending and truncating, \disklog
also stores the old length of the log in \V{previous\_len} field. During
recovery, \disklog first tries to read the log from the disk according to the
length stored in the \V{len} field, compute the checksum, and check it against
the value stored in the \V{checksum} field. If they do not match, \disklog
falls back to use the length stored in the \V{previous\_len} field, computes
the checksum again, and updates the \V{checksum} and \V{len} fields to
reflect the corrected state. The pseudocode for the \disklog protocol is
shown in \Cref{fig:disklog}.

\begin{figure}[htb]
    \centering
    \figcodefont
    \begin{BVerbatim}[commandchars=\\\{\}]
\PY{c}{\PYZsh{} Called by Applier layer after applying log to disk.}
\PY{k}{def} \PY{n+nf}{disklog\PYZus{}truncate}\PY{p}{(}\PY{n}{txn}\PY{p}{)}\PY{p}{:}
    \PY{n}{header} \PY{o}{=} \PY{n}{disk\PYZus{}read}\PY{p}{(}\PY{n}{CommitBlock}\PY{p}{)}
    \PY{n}{header}\PY{o}{.}\PY{n}{previous\PYZus{}len} \PY{o}{=} \PY{n}{header}\PY{o}{.}\PY{n}{len}
    \PY{n}{header}\PY{o}{.}\PY{n}{len} \PY{o}{=} \PY{l+m+mi}{0}
    \PY{n}{disk\PYZus{}write}\PY{p}{(}\PY{n}{CommitBlock}\PY{p}{,} \PY{n}{header}\PY{p}{)}
    \PY{n}{disk\PYZus{}sync}\PY{p}{(}\PY{p}{)}

\PY{c}{\PYZsh{} Called by Applier layer, which guarantees that there\PYZsq{}s enough space.}
\PY{k}{def} \PY{n+nf}{disklog\PYZus{}append}\PY{p}{(}\PY{n}{txn}\PY{p}{)}\PY{p}{:}
    \PY{n}{header} \PY{o}{=} \PY{n}{disk\PYZus{}read}\PY{p}{(}\PY{n}{CommitBlock}\PY{p}{)}
    \PY{n}{write\PYZus{}packed\PYZus{}addresses}\PY{p}{(}\PY{n}{LogDescStart}\PY{p}{,} \PY{n}{header}\PY{o}{.}\PY{n}{len}\PY{p}{,} \PY{n}{txn}\PY{p}{)}
    \PY{n}{pos} \PY{o}{=} \PY{n}{LogDataStart} \PY{o}{+} \PY{n}{header}\PY{o}{.}\PY{n}{len}
    \PY{k}{for} \PY{p}{(}\PY{n}{a}\PY{p}{,} \PY{n}{v}\PY{p}{)} \PY{k}{in} \PY{n}{txn}\PY{o}{.}\PY{n}{iteritems}\PY{p}{(}\PY{p}{)}\PY{p}{:}
        \PY{n}{disk\PYZus{}write}\PY{p}{(}\PY{n}{pos}\PY{p}{,} \PY{n}{v}\PY{p}{)}
        \PY{n}{header}\PY{o}{.}\PY{n}{checksum} \PY{o}{=} \PY{n+nb}{hash}\PY{p}{(}\PY{n}{header}\PY{o}{.}\PY{n}{checksum} \PY{o}{|}\PY{o}{|} \PY{n}{a} \PY{o}{|}\PY{o}{|} \PY{n}{v}\PY{p}{)}
        \PY{n}{pos} \PY{o}{+}\PY{o}{=} \PY{l+m+mi}{1}
    \PY{n}{header}\PY{o}{.}\PY{n}{previous\PYZus{}len} \PY{o}{=} \PY{n}{header}\PY{o}{.}\PY{n}{len}
    \PY{n}{header}\PY{o}{.}\PY{n}{len} \PY{o}{=} \PY{n}{header}\PY{o}{.}\PY{n}{len} \PY{o}{+} \PY{n+nb}{len}\PY{p}{(}\PY{n}{txn}\PY{p}{)}
    \PY{n}{disk\PYZus{}write}\PY{p}{(}\PY{n}{CommitBlock}\PY{p}{,} \PY{n}{header}\PY{p}{)}
    \PY{n}{disk\PYZus{}sync}\PY{p}{(}\PY{p}{)}

\PY{k}{def} \PY{n+nf}{disklog\PYZus{}readlog}\PY{p}{(}\PY{n}{nr}\PY{p}{)}\PY{p}{:}
    \PY{n}{checksum} \PY{o}{=} \PY{n+nb}{hash}\PY{p}{(}\PY{l+m+mi}{0}\PY{p}{)}
    \PY{n}{log} \PY{o}{=} \PY{p}{[}\PY{p}{]}
    \PY{k}{for} \PY{n}{i} \PY{k}{in} \PY{n+nb}{range}\PY{p}{(}\PY{l+m+mi}{0}\PY{p}{,} \PY{n}{nr}\PY{p}{)}\PY{p}{:}
        \PY{n}{a} \PY{o}{=} \PY{n}{read\PYZus{}packed\PYZus{}address}\PY{p}{(}\PY{n}{LogDescStart}\PY{p}{,} \PY{n}{i}\PY{p}{)}
        \PY{n}{v} \PY{o}{=} \PY{n}{disk\PYZus{}read}\PY{p}{(}\PY{n}{LogDataStart} \PY{o}{+} \PY{n}{i}\PY{p}{)}
        \PY{n}{checksum} \PY{o}{=} \PY{n+nb}{hash}\PY{p}{(}\PY{n}{checksum} \PY{o}{|}\PY{o}{|} \PY{n}{a} \PY{o}{|}\PY{o}{|} \PY{n}{v}\PY{p}{)}
        \PY{n}{log}\PY{o}{.}\PY{n}{append}\PY{p}{(}\PY{p}{(}\PY{n}{a}\PY{p}{,} \PY{n}{v}\PY{p}{)}\PY{p}{)}
    \PY{k}{return} \PY{p}{(}\PY{n}{checksum}\PY{p}{,} \PY{n}{log}\PY{p}{)}

\PY{k}{def} \PY{n+nf}{disklog\PYZus{}recover}\PY{p}{(}\PY{p}{)}\PY{p}{:}
    \PY{n}{header} \PY{o}{=} \PY{n}{disk\PYZus{}read}\PY{p}{(}\PY{n}{CommitBlock}\PY{p}{)}
    \PY{p}{(}\PY{n}{checksum}\PY{p}{,} \PY{n}{log}\PY{p}{)} \PY{o}{=} \PY{n}{disklog\PYZus{}readlog}\PY{p}{(}\PY{n}{header}\PY{o}{.}\PY{n}{len}\PY{p}{)} 
    \PY{k}{if} \PY{n}{checksum} \PY{o}{!=} \PY{n}{header}\PY{o}{.}\PY{n}{checksum}\PY{p}{:}
        \PY{p}{(}\PY{n}{checksum}\PY{p}{,} \PY{n}{log}\PY{p}{)} \PY{o}{=} \PY{n}{disklog\PYZus{}readlog}\PY{p}{(}\PY{n}{header}\PY{o}{.}\PY{n}{previous\PYZus{}len}\PY{p}{)} 
        \PY{n}{header}\PY{o}{.}\PY{n}{checksum} \PY{o}{=} \PY{n}{checksum}
        \PY{n}{header}\PY{o}{.}\PY{n}{len} \PY{o}{=} \PY{n}{header}\PY{o}{.}\PY{n}{previous\PYZus{}len}
        \PY{n}{disk\PYZus{}write}\PY{p}{(}\PY{n}{CommitBlock}\PY{p}{,} \PY{n}{header}\PY{p}{)}
        \PY{n}{disk\PYZus{}sync}\PY{p}{(}\PY{p}{)}
    \PY{k}{return} \PY{n}{log}
\end{BVerbatim}

    \caption{Pseudocode of \disklog layer}
    \label{fig:disklog}
\end{figure}


\subsubsection{Formalizing checksums}

A challenge in formalizing and verifying the checksum-based protocol lies
in the probabilistic guarantee that it offers.  In practice, using a strong
collision-resistant hash function (such as SHA-256) ensures that the
probability of a collision is negligible.  Although developers assume that
there are no hash collisions in practice, formalizing this assumption is
difficult.  Theoretically speaking, any hash function (including a
collision-resistant function like SHA-256) has collisions, and as a result,
two different sets of log entries may have the same checksum.
Consequently, stating an axiom that a hash function has no collisions is
unsound and is equivalent to assuming that true is false.  On the other
hand, stating an axiom that a hash function is collision-resistant requires
reasoning about probabilities and the computational power of some
hypothetical adversary that is attempting to find hash collisions.  

An ideal hash model should allows \sys to state specifications in a natural
way---the way that a file system developer might assume---by completely
ignoring the possibility of hash collisions.  Otherwise, All proofs would
have to deal with probabilistic preamble like ``with high probability,
unless there is a hash collision, ...'', and all specification would be
more complex (e.g., after recovery, if there weren't hash collisions during
past crashes, one will have the correct data with high probability).

\paragraph{Approach.}

\syslog has a solution that is both sound and avoids reasoning about the
probability of hash collisions.  The key idea is to treat hash collisions
as function non-termination in the formal semantics of execution.  Recall
from \Cref{sec:chl:model} that any procedure in \chl is composed of
sequences of basic opcodes, such as \cc{disk\_read}, \cc{disk\_write}, etc.
\chl provides a formal semantics for how each of these opcodes should
execute; e.g., a disk read returns the last written value, and a crash
non-deterministically chooses some set of outstanding writes to apply.

\syslog introduces a new opcode to \chl, called \cc{hash}, which computes
the hash value of its input.  \chl's formal semantics keep track of all
inputs ever hashed and their corresponding hash values.  If \cc{hash} is
presented with an input that hashes to the same result as an earlier,
different input, then the \cc{hash} opcode enters an infinite loop and
never returns.

This formalization achieves our goals.  First, it allows \syslog to
conclude that, if \cc{hash} returns the same hash value twice, the inputs
must have been equal (because otherwise \cc{hash} would not have returned),
without reasoning about probabilities.  This allows us to write
specifications about entire file-system operations, such as \cc{rename},
saying that \emph{if the operation returns}, then a transaction must have
been committed.  Second, this formalization is sound, because it does not
prohibit the possibility of hash collisions, and instead, explicitly takes
them into account (by entering an infinite loop on a collision).

At runtime, of course, the \cc{hash} opcode is implemented using a
collision-resistant hash function (SHA-256 in our case).  This hash
function does \emph{not} enter an infinite loop when presented with a
colliding input and, consequently, can differ from the formal semantics of
\cc{hash}.  However, since we know that our hash function is
collision-resistant, we know that the possibility of finding a collision is
negligible, and thus the possibility that the real execution semantics will
differ from the formal one is also negligible.  Consequently, using a
collision-resistant function for \cc{hash} at runtime allows us to capture
the standard assumption made by developers (that hash collisions do not
happen).

\paragraph{Crash safety.}

An additional challenge that arises in \syslog compared to earlier work on
modeling hash collisions~\cite{Barthe:2014:PRV:2578855.2535847} is that the
computer can crash at any point.  This means that, after a crash, the list
of inputs ever hashed can be different from the list of hash inputs in
either the pre- or the post-condition of a procedure.  However, the
file-system recovery code must still be able to reason about the list of
hash inputs after a crash, in order to prove that it recovers the contents
of the on-disk log.  \syslog's solution is to prove that the list of hash
inputs after a crash is a superset of the hash inputs from a procedure's
precondition, which allows \syslog's write-ahead log to prove its
correctness.


\subsection{Log bypass}
\label{sec:log:bypass}

Writes that bypass the log still interact with \sys's write-ahead log.  If
the file system issues a log-bypass write to block $b$, and there is an
un-applied transaction that modified block $b$, it is important that this
transaction does not later overwrite $b$'s contents.  Thus, log-bypass
writes in \sys go through the log abstraction (even though they are not
written to the write-ahead log).

The \cc{dwrite} procedure, exposed by \logapi, \grouplog, and \memlog,
checks if there was a previous logged write to the same address as the
log-bypass write.  In \logapi, previous logged writes to the same address
are discarded (since they have not yet committed).  In \grouplog, if there
are any in-memory transaction writing to the same address, all of them are
flushed to disk.  In \memlog, if the address appears in the on-disk log,
the log is applied, so that a later log apply does not overwrite the block
modified through log bypass.

Another approach could have been to discard previously committed writes to
any block modified via log bypass, even in the lower layers (\grouplog
and \memlog).  However, this approach leads to the same problem that ext4
experienced, where new file blocks can contain data from previously deleted
files after a crash~\cite{git:469017}.  By flushing previously committed
writes, \syslog avoids this problem.

The specification of \syslog's top-level \cc{dwrite} is shown in
\Cref{fig:log_dwrite_ok}.  It differs from the specification of logged
write (see \Cref{fig:log_write_ok}) in three ways. First, \cc{log\_dwrite}
writes directly to the disk; therefore, it changes not only the state of
the current logical disk~(\V{old\_state}), but also all disks from the disk
sequence of the transaction's starting state (\V{old\_disk\_seq}, as they
all derive from the underlying disk state).  Second, much like the
specification for writing to the physical disk, \cc{log\_dwrite} exposes an
asynchronous interface, leaving the updated block in an unsynced
state~(i.e., the block's value-set contains more than one value).
Finally, the crash condition of \cc{log\_dwrite} says that it could recover
into any disk from either the original or the updated disk sequence.  This
is because \cc{log\_dwrite} internally might invoke \cc{log\_flush} to
flush buffered transactions.

\begin{figure}[htb]
\centering
\begin{spec}
    \SPEC  {\PROC{log\_dwrite}{a, v}}
    \PRE   {\PRED{disk}{\F{log\_rep}(\C{ActiveTxn}, \V{old\_disk\_seq}, \V{old\_state})}
            \PRED{old\_state\;\,}{$F * a |-> \valuset{v_0}{vs_0}$}
            \PRED{$\forall$i, old\_disk\_seq[i]\;\,}{$F_i * a |-> \valuset{v_i}{vs_i}$} }
    \POST  {\PRED{disk}{\F{log\_rep}(\C{ActiveTxn}, \V{new\_disk\_seq}, \V{new\_state})}
            \PRED{new\_state}{$F * a |-> \valuset{v}{\vsmerge{v_0}{vs_0}}$}
            \PRED{$\forall$i, new\_disk\_seq[i]}
                 {$F_i * a |-> \valuset{v}{\vsmerge{v_i}{vs_i}}$} }
    \CRASH {\PRED{disk}{\F{would\_recover\_any}(\V{old\_disk\_seq}) $\OR$ \BR
                        \F{would\_recover\_any}(\V{new\_disk\_seq})}}
\end{spec}
\caption{Specification for \logapi's \cc{log\_dwrite}}
\label{fig:log_dwrite_ok}
\end{figure}

The caller of \cc{log\_dwrite} is responsible for syncing the updated block
at the appropriate time.  If the file system is built on top of a write-through
cache, where writes immediately go to the physical disk, the caller can
simply use the disk write barrier (i.e., \cc{disk\_sync}) to persist the
change.  However, \sys is built on top of a write-back cache (see
\Cref{sec:impl}), where writes are buffered in memory until the updated
block is evicted from the cache.

\begin{figure}[htb]
\centering
\begin{spec}
    \SPEC  {\PROC{log\_dsync}{a}}
    \PRE   {\PRED{disk}{\F{log\_rep}(\C{ActiveTxn}, \V{old\_disk\_seq}, \V{old\_state})}
            \PRED{old\_state\;\,}{$F * a |-> \valuset{v_0}{vs_0}$}
            \PRED{$\forall$i, old\_disk\_seq[i]\;\,}{$F_i * a |-> \valuset{v_i}{vs_i}$} }
    \POST  {\PRED{disk}{\F{log\_rep}(\C{ActiveTxn}, \V{new\_disk\_seq}, \V{new\_state})}
            \PRED{new\_state}{$F * a |-> \vsemp{v_0}$}
            \PRED{$\forall$i, new\_disk\_seq[i]}{$F_i * a |-> \vsemp{v_i}$} }
    \CRASH {\PRED{disk}{\F{log\_rep}(\C{ActiveTxn}, \V{old\_disk\_seq}, \V{old\_state})}}
\end{spec}
\caption{Specification for \logapi's \cc{log\_dsync}}
\label{fig:log_dsync_ok}
\end{figure}

To make sure that a block updated through \cc{log\_dwrite} is persisted on
disk, the caller must first evict the block from the buffer cache and then
sync the physical disk.  \syslog offers a \cc{log\_dsync} procedure that
takes care of block syncing in the presence of deferred writes.  The
specification of \cc{log\_dsync} is shown in \Cref{fig:log_dsync_ok}, which
is very similar to \cc{log\_dwrite}'s specification but has a simpler crash
condition.  To improve performance, \syslog also offers a few variants of
\cc{log\_dsync} that allow a caller to evict a list of blocks from the
buffer cache but only issue a single disk sync at the end. \sys uses
\cc{log\_dsync} and its variants to implement \cc{fsync} and
\cc{fdatasync}.

Because log-bypass writes interact subtly with logged writes, this poses
challenges at the file-system level, which uses logged writes to update the
metadata and log-bypass writes to change a file's data.  For example, to
guarantee crash safety, bypassing the log for file data requires that disk
blocks are not reused until the log is flushed to disk.  We discuss how to
address this challenge in the next chapter.


