% $Log: abstract.tex,v $
% Revision 1.1  93/05/14  14:56:25  starflt
% Initial revision
% 
% Revision 1.1  90/05/04  10:41:01  lwvanels
% Initial revision
% 
%
%% The text of your abstract and nothing else (other than comments) goes here.
%% It will be single-spaced and the rest of the text that is supposed to go on
%% the abstract page will be generated by the abstractpage environment.  This
%% file should be \input (not \include 'd) from cover.tex.

Web services like Google, Facebook, and Dropbox are now an essential part of people’s lives. Users
willingly provide their data to these services because these services deliver substantial value in return
through their centralization and analysis of data, such product recommendations and ability to easily
share information. To provide this value, 
these services collect, store, and analyze large amounts of their users’ sensitive
data. However, once the user provides her information to the web service, she loses control over how the
application manipulates that data. For example, a user cannot control where the application forwards
her data. Even if the service wanted to allow users to define access controls, it is unclear how these access
controls should be expressed and enforced. Not only is it difficult to develop these secure access control
mechanisms, but it is also difficult to ensure these mechanisms are practical. In my thesis,
I have built three systems, Splinter, Riverbed, and Veil, to address these concerns.