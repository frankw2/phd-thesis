\section{Conclusion and Future Work}
\label{chap:concl}

In this dissertation, we presented two systems, Veil and Splinter,
to minimize data leakage when a user accesses a web service. 
%Veil is the first system that allows developers to provide
%stronger private browsing semantics without browser-side
%support. Veil minimizes the amount of user data 
%that can be recovered on a shared computer after
%a user terminates her private browsing session.
%Splinter prevents sensitive information from user
%queries in a more practical manner by 
%leveraging a recent cryptographic primitive,
%function secret sharing (FSS). 
These systems have raised new interesting questions,
especially regarding the practicality of 
systems for secure multi-party computation and private queries.

Splinter leveraged
function secret sharing (FSS) to reduce the overhead associated with private queries compared to previous
work. However, open research problems still remain. For example, can private queries scale to billions
or trillions of records? Furthermore, can private queries support more expressive operators like JOINs or
NOT conditions? 

Multi-party computation (MPC) allows participants to evaluate a common function without revealing
their respective inputs. As hospitals, governments, and corporations increasingly transition from
paper records to electronic ones, MPC-style computations represent a natural way to securely enable
cross-organization data sharing. MPC is a well-studied cryptographic solution to this problem, but its
impracticality has prevented widespread adoption. Are there ways to do MPC on more complex
computations, such as graph queries?

As web services have become more prevalent, it is important to not only think about
server-side data leakages but also potential leakage through other channels,
such as the browser and user queries.