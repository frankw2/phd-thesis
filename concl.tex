\section{Conclusion and Future Work}
\label{chap:concl}

In this dissertation, we presented two systems, Veil and Splinter,
to minimize data leakage when a user accesses a web service. 
%Veil is the first system that allows developers to provide
%stronger private browsing semantics without browser-side
%support. Veil minimizes the amount of user data 
%that can be recovered on a shared computer after
%a user terminates her private browsing session.
%Splinter prevents sensitive information from user
%queries in a more practical manner by 
%leveraging a recent cryptographic primitive,
%function secret sharing (FSS). 
These systems have raised new interesting questions,
especially regarding the practicality of 
systems for secure multi-party computation and private queries.

Splinter leveraged
function secret sharing (FSS) to reduce the overhead associated with private queries compared to previous
work. However, open research problems still remain. For example, can private queries scale to billions
or trillions of records? Furthermore, can private queries support more expressive operators like JOINs or
NOT conditions? What are the limitations on the types of FSS can support practically?

Secure multi-party computation (MPC) allows participants to evaluate a common function without revealing
their respective inputs. As hospitals, governments, and corporations increasingly transition from
paper records to electronic ones, MPC-style computations represent a natural way to securely enable
cross-organization data sharing. MPC is a well-studied cryptographic solution to this problem, but its
impracticality has prevented widespread adoption. However, FSS and MPC are closely related, and 
the creation of FSS has shown promise for more practical protocols. In particular,
are there ways to perform MPC practically on more complex computations, such as graph queries?

With the increased number of data and privacy breaches, users are becoming
more aware of the "cost" of using web services, i.e. providing these services
with their data. Moreover, regulations like
GDPR are levying heavy penalties on organizations that do not build
secure mechanisms that protect user data. As a result, we will see
an increase both in users and web services that want more data protections.
In addition, users are willing to trade some performance for privacy and security
as long as it does not affect their experience and workflow. The increased
use of DuckDuckGo and Tor support this trend. Therefore,
as the need for secure systems grows, it is important to focus on performance
and practicality. We hope that the systems in this dissertation serve
as a good foundation for thinking about practical, secure mechanisms
in various web application contexts.