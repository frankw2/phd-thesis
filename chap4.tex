\chapter{Implementation}
\label{sec:implementation}
Our Sieve prototype consists of a Sieve
client, a storage provider daemon, and a
Sieve import daemon that is run by third
parties. Each component is written in Python,
and uses PyCrypto~\cite{pyCrypto} to implement
RSA and AES. For ABE operations, we use the
libfenc~\cite{libfenc} library with elliptic
curves~\cite{mnt224} from the Stanford
Pairing-Based Cryptography library~\cite{pbc}.
To build Sieve's key homomorphic symmetric
cipher~\cite{keyhom}, we use the Ed448-Goldilocks
elliptic curve library~\cite{ed448}.

The storage provider daemon uses BerkeleyDB~\cite{berkdb}
to store encrypted data blocks, and MongoDB~\cite{mongodb}
to store metadata blocks. For each data block,
the key is a GUID, and the value is a symmetrically
encrypted object. For a metadata block, the
key is a set of cleartext ABE attributes, and
the value is an ABE-encrypted GUID and symmetric
key. Metadata blocks are indexed by their
attribute fields, and all metadata blocks for
a particular user are stored in a MongoDB
collection.
% The libfenc library and Ed448-Goldilocks
% elliptic curve libraries are written in C++,
% so whenever we use these libraries,
% we make RPC calls from our Python
% code to call these functions.

The JavaScript code in a web site interacts
with the local Sieve client using a small
RPC library that we provide. When a web site
initially requests access to a user's data,
the site's JavaScript sends an \texttt{XMLHttpRequest} 
to a localhost webserver run by the Sieve
client. The Sieve client then displays a
GUI that allows the user to define an
access policy for the site, and send the
associated ABE key to the site's web server.