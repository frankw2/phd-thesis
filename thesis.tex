% -*- TeX-PDF-mode: t -*-

\newif\ifdraftfooter\draftfootertrue

% draft - Fast compilation, all XXX notes, draft+git footers
%\documentclass[11pt,twoside,draft]{mitthesis}
% (no option) - Full compilation, all XXX notes, draft+git footer
%\documentclass[11pt,twoside]{mitthesis}
% proof - Full compilation, no XXX notes, draft+git footer
%\documentclass[11pt,twoside,final,proof]{mitthesis}
% final - Full compilation, no XXX notes, no footer
\documentclass[11pt,twoside,final]{mitthesis} \draftfooterfalse

\usepackage{pdfsync}
\synctex=1

\usepackage[square,comma,numbers,sort&compress]{natbib}

% Put hyperref in final mode; otherwise hyperlinks are disabled
\usepackage[pdfauthor={Frank Yi-Fei Wang},
            pdftitle={Preventing Data Leakage for Web Service Accesses},
            breaklinks,hidelinks,final,
            bookmarksdepth=3]{hyperref}

% Fonts
\usepackage[T1]{fontenc}
\usepackage[defaultsans]{lato}
% [lf] - Use lining figures by default.  Mostly because I say things
% like "CPU 0" a lot in the text and old style figures make that look
% really weird.
%\usepackage[lf,footnotefigures]{MinionPro} % After amssymb
%\usepackage[toc,bib]{tabfigures}
% No Minion Pro?  Uncomment the next line to use Times
%\usepackage{times,mathptmx}
%\usepackage{amssymb}
%\usepackage{mathrsfs}
%\usepackage{amsfonts}
%\usepackage{textcomp}
\usepackage{verbatim}
\usepackage{amsmath}
\usepackage[charter]{mathdesign}
\usepackage{xspace}
\usepackage{xcolor}
\usepackage{graphicx}
\usepackage{url}
\usepackage{listings}
\usepackage{balance}
\usepackage{subfig}
\usepackage{booktabs}
\usepackage{multirow}
\usepackage{rotating}
\usepackage{fancyvrb}
\usepackage{lastpage}
\usepackage{alltt}
\usepackage{etoolbox}
\usepackage{ifdraft}
\usepackage{titlesec}
\usepackage{cleveref} % After hyperref, listings
\usepackage{fancyhdr}
\usepackage{semantic}
\usepackage{mathtools}
\usepackage{etoolbox}

\IfFileExists{microtype.sty}{
  \usepackage{microtype}
}{
  \PackageError{microtype}{Package microtype not found.  Please
    install texlive-latex-recommended}{}
}

% Local packages below here

\usepackage{xxxnotes}
\usepackage{wordnum}
% Set up fancyhdr before gitinfo
\fancypagestyle{plain}{%
  \fancyhead{}
  \renewcommand{\headrulewidth}{0pt}
}
% If I just \pagestyle{plain}, the left and right footers shift back
% and forth on each page.
\AtBeginDocument{\pagestyle{plain}}
\usepackage{gitinfo}

%\usetikzlibrary{shapes.geometric}

\input{glyphtounicode}
\pdfgentounicode=1

% By default, when including another PDF, rather than including its
% fonts directly, pdfTeX will look for the fonts by name in the fonts
% that it knows and use those (to deduplicate against document fonts).
% Unfortunately, this deduplication has the side-effect of disabling
% pdfTeX's subsetting of the copy of the font included in the PDF.
% This has two consequences.  1) The final PDF gets much bigger.  And
% 2) some PDF engines choke on the complete Minion Pro.  Acroread,
% Evince, and Okular handle it, but at least RepliGo Reader and
% Quickoffice for Android do not (which appear to have a common
% ancestry).
%
% The below directive tells pdfTeX to keep the fonts that come in PDF
% inclusions.  The result is that we get both Type 1 (body) and Type
% 1C (figures) copies of Minion Pro, but the document renders
% correctly.  pdfTeX appears to still merge the Type 1C fonts from the
% various inclusions, so we only wind up with two copies, not several.
% This also makes the final PDF smaller because the sum of the two
% subsetted fonts is smaller than the full Minion Pro.
\pdfinclusioncopyfonts=1

%\DeclareMathAlphabet{\mathcal}{OMS}{cmsy}{m}{n}

% If we're building without Minion Pro, define away \figureversion and
% \sscshape
\ifx\figureversion\undefined%
  \def\figureversion#1{}
\fi
\ifx\sscshape\undefined%
  \let\sscshape=\scshape
\fi

% Draft footer
\ifdraftfooter
\ifoptiondraft{\fancyfoot[L]{\textbf{FAST DRAFT}}}%
{\fancyfoot[L]{\textbf{DRAFT}}}
\fi

% Set page number footer in old style numerals
\fancyfoot[C]{{\figureversion{text}\thepage}}

% One space after periods
\frenchspacing

% Avoid widows and orphans
\widowpenalty=500
\clubpenalty=500

% Double spacing is too big; single spacing is too small
\linespread{1.15}

% Disable annoying inter-paragraph vertical spacing
\raggedbottom

% Bring in margins a bit.  The measure at 11pt on a 5.5" body is
% pretty reasonable (and almost identical to 12pt on a 6" body).
\addtolength{\textwidth}{-0.5in}
\addtolength{\oddsidemargin}{0.25in}
\addtolength{\evensidemargin}{0.25in}
\addtolength{\topmargin}{0.25in}
\addtolength{\textheight}{-0.5in}

% Aggressive figure placement
\renewcommand{\topfraction}{0.9}
\renewcommand{\bottomfraction}{0.8}
\setcounter{topnumber}{2}
\setcounter{bottomnumber}{2}
\setcounter{totalnumber}{4}
\setcounter{dbltopnumber}{2}
\renewcommand{\dbltopfraction}{0.9}
\renewcommand{\textfraction}{0.07}
\renewcommand{\floatpagefraction}{0.7}
\renewcommand{\dblfloatpagefraction}{0.7}

\setlength{\marginparwidth}{0.8in}

% For text from the SOSP paper, define CompactItemize
\newenvironment{CompactItemize}{\begin{itemize}}{\end{itemize}}

% Fix things like \emph{foo\xspace}.  See
% http://tug.org/pipermail/texhax/2006-November/007339.html
\makeatletter
\xspaceaddexceptions{\check@icr}
\makeatother

% Listing style
\lstset{basicstyle=\fontsize{9}{9}\sffamily,showstringspaces=false,columns=fullflexible}

% Use \code{...} to typeset text in code font.  "_"s are accepted in
% the argument of \code.  You can use \code in math mode and you can
% use $math$ in \code.
\makeatletter
\def\code{\protect\@code}
\newlength\code@factor
\setlength\code@factor{0.9pt}
\newlength\code@space
\setlength\code@space{1.2pt}
\def\codefont{\fontsize{\f@size\code@factor}{\f@size\code@factor}\sffamily}
\def\figcodefont{\fontsize{\f@size\code@factor}{\f@size\code@space}\ttfamily}
% Underscore trick derived from http://tex.stackexchange.com/a/146066
\begingroup
% At definition time the token `_` must be active
\catcode`_=\active
\gdef\code@activate@us{%
  \let_\textunderscore
  \catcode`\_=\active
}
\endgroup
\def\@code#1{\ifmmode\text{\@@code{#1}}\else\@@code{#1}\fi}
% Match x-height of 10.95pt Minion Pro to 9.5pt Lato
\def\@@code#1{\begingroup\code@activate@us
  \codefont%
  \scantokens{#1\endinput}%
  \endgroup
}
\makeatother
\let\syscall=\code
% Code in gnuplot labels
\let\gpcode=\code
% Command names
\let\cmd=\code
\let\gpcmd=\gpcode
% Or don't typeset command names differently
%\newcommand{\cmd}[1]{#1}
%\newcommand{\gpcmd}[1]{#1}

\def\UrlFont{\codefont}

\RecustomVerbatimEnvironment{Verbatim}{Verbatim}{formatcom=\footnotesize}


\renewcommand{\ttdefault}{pxtt}

\newcommand{\sys}{FSCQ\@\xspace}
\newcommand{\syslog}{\mbox{\textsc{FscqLog}}\xspace}
\newcommand{\chl}{CHL\@\xspace}

\newcommand{\URL}{\url}
\newcommand{\cc}[1]{\mbox{\code{#1}}}

\newcommand{\defeq}{\mathrel{\Coloneqq}}
\newcommand{\term}[1]{\left<\,\mathrm{#1}\,\right>}
\newcommand{\prog}[1]{\ifmmode\mathbb{#1}\else$\mathbb{#1}$\fi}
\newcommand{\dom}{\mathop{\mathrm{dom}}\,}
\newcommand{\empset}{\varnothing}
\newcommand{\emp}{\mathbf{emp}}
\newcommand{\true}{\mathbf{true}}
\newcommand{\false}{\mathbf{false}}
\newcommand{\ptsto}{\mathrel{\mapsto}}
\newcommand{\subptsto}{\mathrel{\rightarrowtail}}
\newcommand{\sepstar}{~\star~}
\newcommand{\AND}{\ \wedge\ }
\newcommand{\OR}{\ \vee\ }
\newcommand{\Iff}{\ \mathrm{iff}\quad}
\newcommand{\satisfy}{\;\models\;}
\newcommand{\pair}[2]{\langle \mathit{#1}, ~\mathit{#2} \rangle}
\newcommand{\valuset}[2]{\langle #1, ~#2 \rangle}
\newcommand{\vsemp}[1]{\langle #1, ~\empset\rangle}
\newcommand{\vsmerge}[2]{\{#1\} \cup \mathit{#2}}
\newcommand{\app}{+\!\!+}
\newcommand{\cons}[2]{\mathit{#2}\app\,\{\mathit{#1}\}}
\newcommand{\len}[1]{\|\,#1\,\|}

\mathlig{|->}{\ptsto}
\mathlig{|+>}{\subptsto}
\mathlig{*}{\sepstar}
\mathlig{|=}{\satisfy}


% spec formatting

\newcounter{argcount}\newcounter{totalargcount}%
\newcommand{\specargs}[1]{%
  \setcounter{totalargcount}{0}% Reset total count
  \renewcommand*{\do}[1]{\stepcounter{totalargcount}}% Reconfigure count
  \docsvlist{#1}% Count number of items
  \setcounter{argcount}{0}% Reset current item count
  \renewcommand*{\do}[1]{% Reconfigure item \do
    \stepcounter{argcount}% Next item
    \textit{##1}%
    \ifnum\value{argcount}<\value{totalargcount},~\fi% Print item
  }%
  (\docsvlist{#1})% Process list
}

\newcommand{\mlcell}[2][t]{\begin{tabular}[#1]{@{}l@{}}#2\end{tabular}}
\newcommand{\spechdr}[1]{\textsc{#1}}
\newcommand{\specitem}[2]{\spechdr{#1} & \mlcell{#2} \\ }
\newcommand{\specns}[1]{\textbf{#1}}
\newcommand{\specpred}[3][]{%
    \gdef\specpredhdr{\specns{#2}~$\,\models~$}%
    \ifx\\#1\\%
        \gdef\BR{\\\phantom{\specpredhdr}}%
    \else \gdef\BR{\\#1} \fi%
    \specpredhdr#3 \\
}
\newcommand{\specrec}{\Join}
\newcommand{\textpred}[2]{\specns{#1}~$\,\models\;$#2}
\newcommand{\V}[1]{\ifmmode\mathit{#1}\else\textit{#1}\fi} % vars
\newcommand{\F}[1]{\code{#1}}  % function
\newcommand{\C}[1]{\code{#1}}  % constants
\newcommand{\latest}[1]{\V{#1}.\ifmmode\mathrm{latest}\else{latest}\fi}
\newcommand{\PROC}[2]{\code{#1}\,\specargs{#2}}
\newcommand{\SPEC}[2][]{\specitem{SPEC}{%
    #2~%
    \ifx\\#1\\ {}%
    \else $~~\specrec~~$\PROC{#1}{} \fi
}}
\newcommand{\PRE}[1]{\specitem{PRE}{#1}}
\newcommand{\POST}[1]{\specitem{POST}{#1}}
\newcommand{\CRASH}[1]{\specitem{CRASH}{#1}}
\newcommand{\PRED}[3][]{\specpred[#1]{#2}{#3}}
\newcommand{\specfont}{\fontsize{10.5}{14}\selectfont}
\newenvironment{spec}
{\specfont
\begin{tabular}{@{}l@{~ ~}l@{}}}
{\end{tabular}}




\newcommand{\inputcode}[1]{%
  {\small\hrule\vspace{0.5em}\input{#1}\hrule\vspace{0.5em}}}

\newcommand{\inputnodraft}[1]{%
  \ifdraft{#1 omitted in draft mode}{\input{#1}}}

\makeatletter
% XXX I'd much rather strip all periods from *all* captions in the
% list of figures, but I can't for the life of me get TeX to do that.
\def\sc@oneperiod#1.{\@ifnextchar.{\sc@oneperiod #1}{#1.}}
\newcommand{\splitcaption}[2]{%
  \caption[\protect\sc@oneperiod #1.]{#1 #2}}
\makeatother

%\newcommand{\tikzshowbbox}{
%  \path[draw=black] (current bounding box.north west)
%    rectangle (current bounding box.south east);
%  \fill[overlay,black] (0,0) circle (.5mm);
%  \draw[overlay,black] (0,2mm) -- +(0,5mm) (0,-2mm) -- +(0,-5mm)
%                       (2mm,0) -- +(5mm,0) (-2mm,0) -- +(-5mm,0);}

\makeatletter
\newcommand{\asterism}{%
  \noindent\par\vspace{1em}%
  {\hfill $*$ \hspace{3em} $*$ \hspace{3em} $*$ \hfill}\vspace{1em}%
  \par\@afterindentfalse\@afterheading
}
\makeatother

\bibpunct[: ]{[}{]}{,}{n}{XXX}{XXX}

% Do always capitalize "Figure".  Also override the default "Fig."
\crefname{figure}{Figure}{Figures}
\crefname{mysubfigure}{Figure}{Figures}
\Crefname{mysubfigure}{Figure}{Figures}
\newcommand{\thiscref}[1]{this \lcnamecref{#1}}
\newcommand{\Thiscref}[1]{This \lcnamecref{#1}}

% Single-line format with chapter number and a vertical bar.
% \titleformat{\chapter}[hang]{%
%   \huge\bfseries}{\thechapter\hspace{15pt}\rule[-1.2em]{.5pt}{3em}\hspace{15pt}}{0pt}{\huge\bfseries}

% Two-line format with spelled-out chapter number and a horizontal bar
% under the title.
\titleformat{\chapter}[display]%
  {\relax}%
  {\sscshape\Wordnum{\thechapter}}{1em}%
  {\huge\bfseries}[\titlerule]

% Make the TOC only list chapters and sections
\setcounter{tocdepth}{1}

% Make tables use the same numbering and format as figures
\makeatletter
\let\thetable\thefigure
\let\c@table\c@figure
\makeatother

% Fold the one table in to the list of figures
\makeatletter
\renewcommand\ext@table{lof}
\makeatother
\renewcommand\listfigurename{Figures and tables}

\begin{document}


\makeatletter
\def\PY@reset{\let\PY@it=\relax \let\PY@bf=\relax%
    \let\PY@ul=\relax \let\PY@tc=\relax%
    \let\PY@bc=\relax \let\PY@ff=\relax}
\def\PY@tok#1{\csname PY@tok@#1\endcsname}
\def\PY@toks#1+{\ifx\relax#1\empty\else%
    \PY@tok{#1}\expandafter\PY@toks\fi}
\def\PY@do#1{\PY@bc{\PY@tc{\PY@ul{%
    \PY@it{\PY@bf{\PY@ff{#1}}}}}}}
\def\PY#1#2{\PY@reset\PY@toks#1+\relax+\PY@do{#2}}

\expandafter\def\csname PY@tok@gd\endcsname{\def\PY@tc##1{\textcolor[rgb]{0.63,0.00,0.00}{##1}}}
\expandafter\def\csname PY@tok@gu\endcsname{\let\PY@bf=\textbf\def\PY@tc##1{\textcolor[rgb]{0.50,0.00,0.50}{##1}}}
\expandafter\def\csname PY@tok@gt\endcsname{\def\PY@tc##1{\textcolor[rgb]{0.00,0.27,0.87}{##1}}}
\expandafter\def\csname PY@tok@gs\endcsname{\let\PY@bf=\textbf}
\expandafter\def\csname PY@tok@gr\endcsname{\def\PY@tc##1{\textcolor[rgb]{1.00,0.00,0.00}{##1}}}
\expandafter\def\csname PY@tok@cm\endcsname{\let\PY@it=\textit\def\PY@tc##1{\textcolor[rgb]{0.25,0.50,0.50}{##1}}}
\expandafter\def\csname PY@tok@vg\endcsname{\def\PY@tc##1{\textcolor[rgb]{0.10,0.09,0.49}{##1}}}
\expandafter\def\csname PY@tok@mh\endcsname{\def\PY@tc##1{\textcolor[rgb]{0.40,0.40,0.40}{##1}}}
\expandafter\def\csname PY@tok@go\endcsname{\def\PY@tc##1{\textcolor[rgb]{0.53,0.53,0.53}{##1}}}
\expandafter\def\csname PY@tok@ge\endcsname{\let\PY@it=\textit}
\expandafter\def\csname PY@tok@vc\endcsname{\def\PY@tc##1{\textcolor[rgb]{0.10,0.09,0.49}{##1}}}
\expandafter\def\csname PY@tok@il\endcsname{\def\PY@tc##1{\textcolor[rgb]{0.40,0.40,0.40}{##1}}}
\expandafter\def\csname PY@tok@cs\endcsname{\let\PY@it=\textit\def\PY@tc##1{\textcolor[rgb]{0.25,0.50,0.50}{##1}}}
\expandafter\def\csname PY@tok@cp\endcsname{\def\PY@tc##1{\textcolor[rgb]{0.74,0.48,0.00}{##1}}}
\expandafter\def\csname PY@tok@gi\endcsname{\def\PY@tc##1{\textcolor[rgb]{0.00,0.63,0.00}{##1}}}
\expandafter\def\csname PY@tok@gh\endcsname{\let\PY@bf=\textbf\def\PY@tc##1{\textcolor[rgb]{0.00,0.00,0.50}{##1}}}
\expandafter\def\csname PY@tok@ni\endcsname{\let\PY@bf=\textbf\def\PY@tc##1{\textcolor[rgb]{0.60,0.60,0.60}{##1}}}
\expandafter\def\csname PY@tok@nl\endcsname{\def\PY@tc##1{\textcolor[rgb]{0.63,0.63,0.00}{##1}}}
\expandafter\def\csname PY@tok@nn\endcsname{\let\PY@bf=\textbf\def\PY@tc##1{\textcolor[rgb]{0.00,0.00,1.00}{##1}}}
\expandafter\def\csname PY@tok@no\endcsname{\def\PY@tc##1{\textcolor[rgb]{0.53,0.00,0.00}{##1}}}
\expandafter\def\csname PY@tok@na\endcsname{\def\PY@tc##1{\textcolor[rgb]{0.49,0.56,0.16}{##1}}}
\expandafter\def\csname PY@tok@nb\endcsname{\def\PY@tc##1{\textcolor[rgb]{0.00,0.50,0.00}{##1}}}
\expandafter\def\csname PY@tok@nc\endcsname{\let\PY@bf=\textbf\def\PY@tc##1{\textcolor[rgb]{0.00,0.00,1.00}{##1}}}
\expandafter\def\csname PY@tok@nd\endcsname{\def\PY@tc##1{\textcolor[rgb]{0.67,0.13,1.00}{##1}}}
\expandafter\def\csname PY@tok@ne\endcsname{\let\PY@bf=\textbf\def\PY@tc##1{\textcolor[rgb]{0.82,0.25,0.23}{##1}}}
\expandafter\def\csname PY@tok@nf\endcsname{\def\PY@tc##1{\textcolor[rgb]{0.00,0.00,1.00}{##1}}}
\expandafter\def\csname PY@tok@si\endcsname{\let\PY@bf=\textbf\def\PY@tc##1{\textcolor[rgb]{0.73,0.40,0.53}{##1}}}
\expandafter\def\csname PY@tok@s2\endcsname{\def\PY@tc##1{\textcolor[rgb]{0.73,0.13,0.13}{##1}}}
\expandafter\def\csname PY@tok@vi\endcsname{\def\PY@tc##1{\textcolor[rgb]{0.10,0.09,0.49}{##1}}}
\expandafter\def\csname PY@tok@nt\endcsname{\let\PY@bf=\textbf\def\PY@tc##1{\textcolor[rgb]{0.00,0.50,0.00}{##1}}}
\expandafter\def\csname PY@tok@nv\endcsname{\def\PY@tc##1{\textcolor[rgb]{0.10,0.09,0.49}{##1}}}
\expandafter\def\csname PY@tok@s1\endcsname{\def\PY@tc##1{\textcolor[rgb]{0.73,0.13,0.13}{##1}}}
\expandafter\def\csname PY@tok@kd\endcsname{\let\PY@bf=\textbf\def\PY@tc##1{\textcolor[rgb]{0.00,0.50,0.00}{##1}}}
\expandafter\def\csname PY@tok@sh\endcsname{\def\PY@tc##1{\textcolor[rgb]{0.73,0.13,0.13}{##1}}}
\expandafter\def\csname PY@tok@sc\endcsname{\def\PY@tc##1{\textcolor[rgb]{0.73,0.13,0.13}{##1}}}
\expandafter\def\csname PY@tok@sx\endcsname{\def\PY@tc##1{\textcolor[rgb]{0.00,0.50,0.00}{##1}}}
\expandafter\def\csname PY@tok@bp\endcsname{\def\PY@tc##1{\textcolor[rgb]{0.00,0.50,0.00}{##1}}}
\expandafter\def\csname PY@tok@c1\endcsname{\let\PY@it=\textit\def\PY@tc##1{\textcolor[rgb]{0.25,0.50,0.50}{##1}}}
\expandafter\def\csname PY@tok@kc\endcsname{\let\PY@bf=\textbf\def\PY@tc##1{\textcolor[rgb]{0.00,0.50,0.00}{##1}}}
\expandafter\def\csname PY@tok@c\endcsname{\let\PY@it=\textit\def\PY@tc##1{\textcolor[rgb]{0.25,0.50,0.50}{##1}}}
\expandafter\def\csname PY@tok@mf\endcsname{\def\PY@tc##1{\textcolor[rgb]{0.40,0.40,0.40}{##1}}}
\expandafter\def\csname PY@tok@err\endcsname{\def\PY@bc##1{\setlength{\fboxsep}{0pt}\fcolorbox[rgb]{1.00,0.00,0.00}{1,1,1}{\strut ##1}}}
\expandafter\def\csname PY@tok@mb\endcsname{\def\PY@tc##1{\textcolor[rgb]{0.40,0.40,0.40}{##1}}}
\expandafter\def\csname PY@tok@ss\endcsname{\def\PY@tc##1{\textcolor[rgb]{0.10,0.09,0.49}{##1}}}
\expandafter\def\csname PY@tok@sr\endcsname{\def\PY@tc##1{\textcolor[rgb]{0.73,0.40,0.53}{##1}}}
\expandafter\def\csname PY@tok@kn\endcsname{\let\PY@bf=\textbf\def\PY@tc##1{\textcolor[rgb]{0.00,0.50,0.00}{##1}}}
\expandafter\def\csname PY@tok@gp\endcsname{\let\PY@bf=\textbf\def\PY@tc##1{\textcolor[rgb]{0.00,0.00,0.50}{##1}}}
\expandafter\def\csname PY@tok@kr\endcsname{\let\PY@bf=\textbf\def\PY@tc##1{\textcolor[rgb]{0.00,0.50,0.00}{##1}}}
\expandafter\def\csname PY@tok@s\endcsname{\def\PY@tc##1{\textcolor[rgb]{0.73,0.13,0.13}{##1}}}
\expandafter\def\csname PY@tok@kp\endcsname{\def\PY@tc##1{\textcolor[rgb]{0.00,0.50,0.00}{##1}}}
\expandafter\def\csname PY@tok@w\endcsname{\def\PY@tc##1{\textcolor[rgb]{0.73,0.73,0.73}{##1}}}
\expandafter\def\csname PY@tok@ow\endcsname{\let\PY@bf=\textbf\def\PY@tc##1{\textcolor[rgb]{0.67,0.13,1.00}{##1}}}
\expandafter\def\csname PY@tok@sb\endcsname{\def\PY@tc##1{\textcolor[rgb]{0.73,0.13,0.13}{##1}}}
\expandafter\def\csname PY@tok@k\endcsname{\let\PY@bf=\textbf\def\PY@tc##1{\textcolor[rgb]{0.00,0.50,0.00}{##1}}}
\expandafter\def\csname PY@tok@se\endcsname{\let\PY@bf=\textbf\def\PY@tc##1{\textcolor[rgb]{0.73,0.40,0.13}{##1}}}
\expandafter\def\csname PY@tok@sd\endcsname{\let\PY@it=\textit\def\PY@tc##1{\textcolor[rgb]{0.73,0.13,0.13}{##1}}}

\def\PYZbs{\char`\\}
\def\PYZus{\char`\_}
\def\PYZob{\char`\{}
\def\PYZcb{\char`\}}
\def\PYZca{\char`\^}
\def\PYZam{\char`\&}
\def\PYZlt{\char`\<}
\def\PYZgt{\char`\>}
\def\PYZsh{\char`\#}
\def\PYZpc{\char`\%}
\def\PYZdl{\char`\$}
\def\PYZhy{\char`\-}
\def\PYZsq{\char`\'}
\def\PYZdq{\char`\"}
\def\PYZti{\char`\~}
% for compatibility with earlier versions
\def\PYZat{@}
\def\PYZlb{[}
\def\PYZrb{]}
\makeatother



% -*- TeX-master: "paper.tex"; TeX-PDF-mode: t -*-

\title{Preventing Data Leakage in Web Services}
\author{Frank Yi-Fei Wang}
\prevdegrees{
  B.S., Stanford University (2012) \\
  S.M., Massachusetts Institute of Technology (2016) }
\department{Department of Electrical Engineering and Computer Science}
\degree{Doctor of Philosophy in Computer Science}
\degreemonth{September}
\degreeyear{2018}
\thesisdate{\today}
\supervisor{Nickolai Zeldovich}{Professor}
\supervisor{James Mickens}{Associate Professor}
\chairman{Leslie A. Kolodziejski}{Professor of Electrical Engineering and Computer Science \\ 
	Chair, Department Committee on Graduate Students}
\maketitle

\cleardoublepage

\begin{abstractpage}
Web services, like Google, Facebook, and Dropbox, 
are a regular part of users' lives. 
As a form of payment, these services
collect, store, and analyze user data.
Even accessing these web services can leak a substantial
amount of data. 

This dissertation presents two practical,
secure systems, Veil and Splinter, that
prevents some of this data leakage. Veil minimizes
client-side information leakage from the browser by allowing
web page developers to enforce stronger private browsing 
semantics without browser support. Splinter protects
sensitive information present in a users' query on cleartext datasets
in a practical manner by leveraging a recent cryptographic
primitive, Function Secret Sharing (FSS).
\end{abstractpage}
\cleardoublepage

\begin{singlespace}
\chapter*{Acknowledgments}

More acknowledgments here.

\asterism

%This research was supported in part by NSF awards CNS-1053143 and
%CCF-1253229, and by the CSAIL cybersecurity initiative.

This work was
partially supported by an NSF Graduate Research Fellowship
(Grant No. 2013135952) and by NSF awards
CNS-1053143, CNS-1413920, CNS-1350619, 
and CNS-1414119. \\

The dissertation incorporates and extends work published
in the following papers:

\begin{quote}

  Frank Wang, James Mickens, and Nickolai Zeldovich.
  \newblock Veil: Private Browsing Semantics Without Browser-side Assistance.
  \newblock In \emph{Proceedings of Network and Distributed System Security Symposium (\mbox{NDSS})},
  2018.
  
  Frank Wang, Catherine Yun, Shafi Goldwasser, Vinod Vaikuntanathan, and Matei Zaharia.
  \newblock Splinter: Practical Private Queries on Public Data.
  \newblock In \emph{Proceedings of Networked Systems Design
  	and Implementation (\mbox{NSDI})}, 2017.
  
\end{quote}

\end{singlespace}

\pagestyle{plain}
\tableofcontents
\newpage
\listoffigures

% The preamble folded tables in to the list of figures
% \newpage
% \listoftables

% Shift sectioning commands to start at \chapter
\let\subsubsubsection\subsubsection
\let\subsubsection\subsection
\let\subsection\section
\let\section\chapter

%\section{Introduction}
\label{chap:intro}

\subsection{Motivation}
Consumers are increasing their usage of web services. Whenever
users interact with these applications, the web service collects
their data, which can contain sensitive 
information, such as their medical conditions,
political preferences, and income~\cite{narayanan2010myths, narayanan2008robust}.
Unfortunately, this data can 
leak through numerous channels on both the server
and client. As web systems become more complex, the 
number of these channels continues to grow. More specifically,
on the server side, data can leak through data breaches
or be accessed by malicious service administrators. On the client side,
sensitive user information can leak from the browser whenever
a user accesses a web application. These scenarios raise an important
question. How do we build secure
mechanisms to prevent this data leakage?

Much research has focused on building systems~\cite{popa:mylar, popa:cryptdb, li:sundr, feldman:sporc} 
that protect sensitive user data stored in web application databases
or cloud providers
from leaking as a result of data breaches or malicious service
providers. However, the focus of
this dissertation is different. We are not protecting sensitive 
user data that already exists on the server.
Instead, we build systems
that prevent data leakage on the client (Veil), specifically
from the browser, and that protect sensitive data in queries (Splinter).
In the section below, we provide an overview of these two systems.

\subsection{Our Systems}

\subsubsection{Veil: Private Browsing Semantics without Browser-side Assistance}
All popular web browsers offer a ``private browsing
mode.'' After a private session terminates, the
browser is supposed to remove client-side
evidence that the session occurred. Unfortunately,
browsers still leak information through the file
system, the browser cache, the DNS cache, and
on-disk reflections of RAM such as the swap file.

Veil is a new deployment framework that allows
web developers to prevent these information leaks,
or at least reduce their likelihood. Veil leverages
the fact that, even though developers do not control the
client-side browser implementation, developers do
control 1) the content that is sent to those browsers,
and 2) the servers which deliver that content.
Veil web sites collectively store their content
on Veil's \emph{blinding servers} instead
of on individual, site-specific servers. To publish
a new page, developers pass their HTML, CSS, and
JavaScript files to Veil's compiler; the compiler
transforms the URLs in the content so that, when
the page loads on a user's browser, URLs are derived
from a secret user key. The blinding service and
the Veil page exchange encrypted
data that is also protected by the user's key. The
result is that Veil pages can safely store encrypted
content in the browser cache; furthermore, the URLs exposed
to system interfaces like the DNS cache are
unintelligible to attackers who do not possess
the user's key. To protect against post-session
inspection of swap file artifacts, Veil uses
heap walking (which minimizes the likelihood that
secret data is paged out), content mutation (which
garbles in-memory artifacts if they do get swapped out),
and DOM hiding (which prevents the browser from
learning site-specific HTML, CSS, and JavaScript
content in the first place). Veil pages load on unmodified
commodity browsers, allowing developers to provide
stronger semantics for private browsing without forcing
users to install or reconfigure their machines. Veil
provides these guarantees even if the user does not
visit a page using a browser's native privacy mode;
indeed, Veil's protections are \emph{stronger}
than what the browser alone can provide.

\subsubsection{Splinter: Practical, Private Web Application Queries}

Many online services let users query datasets such as maps, flight prices, patents,
and medical information. The datasets themselves do not contain sensitive user information,
but unfortunately, users' queries on these datasets can reveal sensitive information.
This dissertation presents Splinter, a system that protects users' queries and
scales to realistic applications.
A user splits her query into multiple parts and sends each part 
to a different provider that holds a copy of the data.
As long as any one of the providers is honest and does not collude with the
others, the providers cannot determine the query.
Splinter uses and extends a new cryptographic primitive called Function Secret Sharing (FSS) 
that makes it up to an order of magnitude more efficient than prior systems based on 
other cryptographic techniques such as Private Information Retrieval and garbled circuits.
We develop protocols extending FSS to new types of queries, such as MAX and TOPK queries. 
We also provide an optimized implementation of FSS using AES-NI instructions and multicores.
Splinter achieves end-to-end latencies below 1.6 seconds for realistic workloads 
including a Yelp clone, flight search, and map routing.

%\subsubsection{Dissertation Roadmap}
%
%The dissertation will be organization like the following: 
%Chapter~\ref{chap:veil} and Chapter~\ref{chap:splinter} will motivate and describe Veil
%and Splinter respectively. Chapter~\ref{chap:concl} will describe future work.

%\input{relwk}
%\input{bkgrd}
%\input{chl}
%\begin{BVerbatim}[commandchars=\\\{\}]
\PY{c}{\PYZsh{} Called by Applier layer after applying log to disk.}
\PY{k}{def} \PY{n+nf}{disklog\PYZus{}truncate}\PY{p}{(}\PY{n}{txn}\PY{p}{)}\PY{p}{:}
    \PY{n}{header} \PY{o}{=} \PY{n}{disk\PYZus{}read}\PY{p}{(}\PY{n}{CommitBlock}\PY{p}{)}
    \PY{n}{header}\PY{o}{.}\PY{n}{previous\PYZus{}len} \PY{o}{=} \PY{n}{header}\PY{o}{.}\PY{n}{len}
    \PY{n}{header}\PY{o}{.}\PY{n}{len} \PY{o}{=} \PY{l+m+mi}{0}
    \PY{n}{disk\PYZus{}write}\PY{p}{(}\PY{n}{CommitBlock}\PY{p}{,} \PY{n}{header}\PY{p}{)}
    \PY{n}{disk\PYZus{}sync}\PY{p}{(}\PY{p}{)}

\PY{c}{\PYZsh{} Called by Applier layer, which guarantees that there\PYZsq{}s enough space.}
\PY{k}{def} \PY{n+nf}{disklog\PYZus{}append}\PY{p}{(}\PY{n}{txn}\PY{p}{)}\PY{p}{:}
    \PY{n}{header} \PY{o}{=} \PY{n}{disk\PYZus{}read}\PY{p}{(}\PY{n}{CommitBlock}\PY{p}{)}
    \PY{n}{write\PYZus{}packed\PYZus{}addresses}\PY{p}{(}\PY{n}{LogDescStart}\PY{p}{,} \PY{n}{header}\PY{o}{.}\PY{n}{len}\PY{p}{,} \PY{n}{txn}\PY{p}{)}
    \PY{n}{pos} \PY{o}{=} \PY{n}{LogDataStart} \PY{o}{+} \PY{n}{header}\PY{o}{.}\PY{n}{len}
    \PY{k}{for} \PY{p}{(}\PY{n}{a}\PY{p}{,} \PY{n}{v}\PY{p}{)} \PY{k}{in} \PY{n}{txn}\PY{o}{.}\PY{n}{iteritems}\PY{p}{(}\PY{p}{)}\PY{p}{:}
        \PY{n}{disk\PYZus{}write}\PY{p}{(}\PY{n}{pos}\PY{p}{,} \PY{n}{v}\PY{p}{)}
        \PY{n}{header}\PY{o}{.}\PY{n}{checksum} \PY{o}{=} \PY{n+nb}{hash}\PY{p}{(}\PY{n}{header}\PY{o}{.}\PY{n}{checksum} \PY{o}{|}\PY{o}{|} \PY{n}{a} \PY{o}{|}\PY{o}{|} \PY{n}{v}\PY{p}{)}
        \PY{n}{pos} \PY{o}{+}\PY{o}{=} \PY{l+m+mi}{1}
    \PY{n}{header}\PY{o}{.}\PY{n}{previous\PYZus{}len} \PY{o}{=} \PY{n}{header}\PY{o}{.}\PY{n}{len}
    \PY{n}{header}\PY{o}{.}\PY{n}{len} \PY{o}{=} \PY{n}{header}\PY{o}{.}\PY{n}{len} \PY{o}{+} \PY{n+nb}{len}\PY{p}{(}\PY{n}{txn}\PY{p}{)}
    \PY{n}{disk\PYZus{}write}\PY{p}{(}\PY{n}{CommitBlock}\PY{p}{,} \PY{n}{header}\PY{p}{)}
    \PY{n}{disk\PYZus{}sync}\PY{p}{(}\PY{p}{)}

\PY{k}{def} \PY{n+nf}{disklog\PYZus{}readlog}\PY{p}{(}\PY{n}{nr}\PY{p}{)}\PY{p}{:}
    \PY{n}{checksum} \PY{o}{=} \PY{n+nb}{hash}\PY{p}{(}\PY{l+m+mi}{0}\PY{p}{)}
    \PY{n}{log} \PY{o}{=} \PY{p}{[}\PY{p}{]}
    \PY{k}{for} \PY{n}{i} \PY{k}{in} \PY{n+nb}{range}\PY{p}{(}\PY{l+m+mi}{0}\PY{p}{,} \PY{n}{nr}\PY{p}{)}\PY{p}{:}
        \PY{n}{a} \PY{o}{=} \PY{n}{read\PYZus{}packed\PYZus{}address}\PY{p}{(}\PY{n}{LogDescStart}\PY{p}{,} \PY{n}{i}\PY{p}{)}
        \PY{n}{v} \PY{o}{=} \PY{n}{disk\PYZus{}read}\PY{p}{(}\PY{n}{LogDataStart} \PY{o}{+} \PY{n}{i}\PY{p}{)}
        \PY{n}{checksum} \PY{o}{=} \PY{n+nb}{hash}\PY{p}{(}\PY{n}{checksum} \PY{o}{|}\PY{o}{|} \PY{n}{a} \PY{o}{|}\PY{o}{|} \PY{n}{v}\PY{p}{)}
        \PY{n}{log}\PY{o}{.}\PY{n}{append}\PY{p}{(}\PY{p}{(}\PY{n}{a}\PY{p}{,} \PY{n}{v}\PY{p}{)}\PY{p}{)}
    \PY{k}{return} \PY{p}{(}\PY{n}{checksum}\PY{p}{,} \PY{n}{log}\PY{p}{)}

\PY{k}{def} \PY{n+nf}{disklog\PYZus{}recover}\PY{p}{(}\PY{p}{)}\PY{p}{:}
    \PY{n}{header} \PY{o}{=} \PY{n}{disk\PYZus{}read}\PY{p}{(}\PY{n}{CommitBlock}\PY{p}{)}
    \PY{p}{(}\PY{n}{checksum}\PY{p}{,} \PY{n}{log}\PY{p}{)} \PY{o}{=} \PY{n}{disklog\PYZus{}readlog}\PY{p}{(}\PY{n}{header}\PY{o}{.}\PY{n}{len}\PY{p}{)} 
    \PY{k}{if} \PY{n}{checksum} \PY{o}{!=} \PY{n}{header}\PY{o}{.}\PY{n}{checksum}\PY{p}{:}
        \PY{p}{(}\PY{n}{checksum}\PY{p}{,} \PY{n}{log}\PY{p}{)} \PY{o}{=} \PY{n}{disklog\PYZus{}readlog}\PY{p}{(}\PY{n}{header}\PY{o}{.}\PY{n}{previous\PYZus{}len}\PY{p}{)} 
        \PY{n}{header}\PY{o}{.}\PY{n}{checksum} \PY{o}{=} \PY{n}{checksum}
        \PY{n}{header}\PY{o}{.}\PY{n}{len} \PY{o}{=} \PY{n}{header}\PY{o}{.}\PY{n}{previous\PYZus{}len}
        \PY{n}{disk\PYZus{}write}\PY{p}{(}\PY{n}{CommitBlock}\PY{p}{,} \PY{n}{header}\PY{p}{)}
        \PY{n}{disk\PYZus{}sync}\PY{p}{(}\PY{p}{)}
    \PY{k}{return} \PY{n}{log}
\end{BVerbatim}

%\input{spec}
%\input{impl}
%\input{eval}
%\section{Conclusion and Future Work}
\label{chap:concl}

In this dissertation, we presented two systems, Veil and Splinter,
to minimize data leakage when a user accesses a web service. 
%Veil is the first system that allows developers to provide
%stronger private browsing semantics without browser-side
%support. Veil minimizes the amount of user data 
%that can be recovered on a shared computer after
%a user terminates her private browsing session.
%Splinter prevents sensitive information from user
%queries in a more practical manner by 
%leveraging a recent cryptographic primitive,
%function secret sharing (FSS). 
These systems have raised new interesting questions,
especially regarding the practicality of 
systems for secure multi-party computation and private queries.

Splinter leveraged
function secret sharing (FSS) to reduce the overhead associated with private queries compared to previous
work. However, open research problems still remain. For example, can private queries scale to billions
or trillions of records? Furthermore, can private queries support more expressive operators like JOINs or
NOT conditions? What are the limitations on the types of FSS can support practically?

Multi-party computation (MPC) allows participants to evaluate a common function without revealing
their respective inputs. As hospitals, governments, and corporations increasingly transition from
paper records to electronic ones, MPC-style computations represent a natural way to securely enable
cross-organization data sharing. MPC is a well-studied cryptographic solution to this problem, but its
impracticality has prevented widespread adoption. Are there ways to do MPC on more complex
computations, such as graph queries?

With the increased number of data and privacy breaches, users are becoming
more aware of the "cost" of using web services. Moreover, regulations like
GDPR are levying heavy penalties on organizations that do not build
secure mechanisms that protect user data. As a result, we will see
an increase both in users and web services that want to prevent data leakage.
We hope that the systems in this dissertation serve
as a good foundation for thinking about practical, secure mechanisms
that prevent this leakage.

{
\begin{singlespace}
% Since we're using natbib in numbers mode, we don't need plainnat,
% which exists to feed authors and years back in to natbib.  As a
% result, it complains about entries without years, which we don't
% care about.
%\bibliographystyle{plainnat}
\bibliographystyle{plain}
\bibliography{n-str,p,n,n-conf}
\end{singlespace}
}

\end{document}
